\documentclass[a4paper,14pt]{extarticle}

\usepackage[T2A]{fontenc}
\usepackage[utf8]{inputenc}
\usepackage[english, russian]{babel}

\usepackage[left=30mm, right=10mm, top=20mm, bottom=20mm]{geometry}

\usepackage{tempora}
\usepackage{setspace}
\onehalfspacing

\usepackage{titlesec}
\titleformat{\section}[block]{\bfseries\centering\MakeUppercase}{\thesection.}{1em}{}
\titleformat{\subsection}[block]{\bfseries}{\thesubsection.}{1em}{}
\titleformat{\subsubsection}[block]{\bfseries}{\thesubsubsection.}{1em}{}

\renewcommand{\contentsname}{\hfill \textbf{СОДЕРЖАНИЕ} \hfill\null}

\usepackage{indentfirst}
\setlength{\parindent}{1.25cm}

\usepackage{amsmath, amsfonts, amssymb}
\usepackage{graphicx}
\usepackage{caption}
\usepackage{subcaption}
\usepackage{float}
\usepackage{tikz}
\usetikzlibrary{patterns}
\usepackage{cmap}
\usepackage{hyperref}
\usepackage{xcolor}
\usepackage{listings}

\definecolor{LightGray}{gray}{0.7}

\lstdefinestyle{code}{
    language=matlab, % change if needed
    basicstyle=\small\ttfamily,
    numbers=left,
    numberstyle=\small\color{LightGray},
    stepnumber=1,
    numbersep=5pt,
    backgroundcolor=\color{white},
    showspaces=false,
    showstringspaces=false,
    showtabs=false,
    tabsize=4,
    captionpos=b,
    breaklines=true,
    breakatwhitespace=false,
    frame=single,
    rulecolor=\color{LightGray},
    linewidth=\linewidth,
    keywordstyle=\color{blue}\bfseries,
    commentstyle=\color{green!40!black},
    stringstyle=\color{violet},
    escapeinside={\%*}{*)},
    xleftmargin=10pt,
    xrightmargin=10pt,
    framexleftmargin=0pt,
    framexrightmargin=0pt
}
\lstset{style=code}

\hypersetup{
    colorlinks=true,
    linkcolor=blue,
    filecolor=magenta,
    urlcolor=cyan,
    pdftitle={lab arc},
    pdfauthor={Rumyantsev Alexey},
    pdfsubject={control},
    pdfkeywords={LaTeX, PDF},
    pdfpagemode=FullScreen,
}

\graphicspath{{src/images/}}

\begin{document}

\begin{titlepage}
    \begin{center}
        МИНИСТЕРСТВО НАУКИ И ВЫСШЕГО ОБРАЗОВАНИЯ РОССИЙСКОЙ ФЕДЕРАЦИИ\\
        \vspace*{2.5mm}
        Федеральное государственное автономное образовательное учреждение высшего образования
        «НАЦИОНАЛЬНЫЙ ИССЛЕДОВАТЕЛЬСКИЙ УНИВЕРСИТЕТ ИТМО»\\
        \vspace*{2.5mm}
        ФАКУЛЬТЕТ СИСТЕМ УПРАВЛЕНИЯ И РОБОТОТЕХНИКИ
        \vfill

        {\large\bfseries ОТЧЕТ ПО ЛАБОРАТОРНОЙ РАБОТЕ №1}\\
        {\large по дисциплине}\\
        {\large«ТЕОРИЯ ОПТИМАЛЬНОГО УПРАВЛЕНИЯ»}\\
        {\large на тему}\\
        {\large «ПОИСК МИНИМУМА С ПОМОЩЬЮ МЕТОДОВ СТАТИЧЕСКОЙ ОПТИМИЗАЦИИ»}\\
        Вариант 31
        \vfill

        \begin{flushright}
            Выполнил: студент гр. R3441\\
            Румянцев А. А.\medskip\\

            Проверил: преподаватель\\
            Парамонов А. В.
        \end{flushright}
        \vfill

        Санкт-Петербург\\
        2025
    \end{center}
\end{titlepage}

\setcounter{page}{2}
\tableofcontents
\newpage


\section{Цель работы}
Исследовать методы статической оптимизации
и найти минимум критерия качества $J(x,u)$.


\section{Постановка задачи}
Дан критерий качества $J(x,u)$ и ограничение $c(x,u)$.
Необходимо:
\begin{enumerate}
    \item Найти глобальный минимум $J(x,u)$ на основе необходимого и достаточного условий экстремума:
    \begin{enumerate}
        \item Без ограничений;
        \item С ограничением в виде равенства $c(x,u)=0$;
        \item С ограничением в виде неравенства $c(x,u)\leq0$.
    \end{enumerate}
    \item Осуществить градиентный поиск минимума критерия качества $J_1(x,u)=J(x,u)$:
    \begin{enumerate}
        \item Методом Ньютона-Рафсона произвести пошаговый расчет экстремума.
        \item Методом наискорейшего спуска для двух различных $\gamma$
        (соответствующих колебательной и апериодической сходимостям) произвести
        пошаговый расчет экстремума.
    \end{enumerate}    
\end{enumerate}


\section{Экспериментальная часть}
\subsection{Исходные данные}
Согласно варианту 31, критерий качества:
$$
J(x,u)=4x^2+3u^2+6xu+9x+2u-7
$$
Ограничение:
$$
c(x,u)=8x^2+7u+2
$$


\subsection{Поиск глобального минимума}
\subsubsection{Без ограничений}
Градиент критерия:
$$
\operatorname{grad}{J(x,u)}=\nabla J(x,u)=\begin{bmatrix}
    \frac{\partial J(x,u)}{\partial x}\\ \frac{\partial J(x,u)}{\partial u}
\end{bmatrix}=\begin{bmatrix}
    8x+6u+9\\ 6x+6u+2
\end{bmatrix}
$$
Приравняем производные к нулю и найдем $x,u$:
$$
\begin{cases}
    8x+6u+9=0,\\ 6x+6u+2=0
\end{cases}\Rightarrow \begin{cases}
    x=-7/2,\\u=19/6
\end{cases}
$$
Вычислим значение критерия:
$$
J\left(-\frac{7}{2},\frac{19}{6}\right)=4\left( -\frac{7}{2} \right)^2+
3\left( \frac{19}{6} \right)^2+
6\left( -\frac{7}{2} \right)\left( \frac{19}{6} \right)+
9\left( -\frac{7}{2} \right)+2\left( \frac{19}{6} \right)-7,
$$
$$
J\left(-\frac{7}{2},\frac{19}{6}\right)\approx-19.583
$$
Таким образом, для точки $\left(-7/2,19/6\right)$
значение критерия $J\approx-19.583$ -- глобальный минимум без ограничений.


\subsubsection{С ограничением в виде равенства}
Ограничение: $c(x,u)=8x^2+7u+2=0$.


Функция Лагранжа:
$$
L(x,u,\lambda)=J(x,u)+\lambda c(x,u),
$$
где $\lambda$ -- множитель Лагранжа.


Подставим:
$$
L(x,u,\lambda)=4x^2+3u^2+6xu+9x+2u-7+\lambda\left( 8x^2+7u+2 \right)
$$
Система частных производных лагранжиана:
$$
\begin{cases}
    \frac{\partial L}{\partial \lambda}=c(x,u)=8x^2+7u+2=0,\\
    \frac{\partial L}{\partial x}=8x+6u+9+16x\lambda=0,\\
    \frac{\partial L}{u}=6x+6u+2+7\lambda=0
\end{cases}
$$
Выразим $\lambda$:
$$
\lambda=-\frac{6x+6u+2}{7}
$$
Подставим в $\partial L/\partial x$:
$$
8x+6u+9-16x\left( \frac{6x+6u+2}{7} \right)=0,
$$
$$
7\left( 8x+6u+9 \right)-16x\left( 6x+6u+2 \right)=0,
$$
$$
-96x^2-96xu+24x+42u+63=0
$$
Выразим $u$ из $\partial L/\partial \lambda$:
$$
u=-\frac{8x^2+2}{7}
$$
Подставим:
$$
-96x^2+96x\left( \frac{8x^2+2}{7} \right)+24x-42\left( \frac{8x^2+2}{7} \right)+63=0,
$$
$$
256x^3-336x^2+120x+119=0\Rightarrow\begin{cases}
    x_1\approx-0.40167,\\
    x_{2,3}\approx0.85708\pm0.65014i
\end{cases}
$$
В реальных задачах комплексные решения лагранжевых уравнений отбрасываются.


Подставим $x_1=-0.40167$ в $u$:
$$
u=-\frac{8\left( -0.40167 \right)^2+2}{7}\approx-0.4701
$$
Вычислим $J$ при $x=-0.40167,u=-0.4701$:
\begin{align*}
    J(-0.40167,-0.4701)= &4(-0.40167)^2+3(-0.4701)^2+\\
    &+6(-0.40167)(-0.4701)+9(-0.40167)+\\
    &+2(-0.4701)-7=\approx -9.11394
\end{align*}


Так как реальное решение единственное,
то единственным кандидатом на экстремум на границе $c(x,u)=0$ является точка $(-0.40167,-0.4701)$,
значение критерия в которой $J\approx-9.11394$ -- глобальный минимум с ограничением в виде равенства.


\subsubsection{С ограничением в виде неравенства}
Ограничение: $c(x,u)=8x^2+7u+2\leq0$.


Условие Куна-Таккера: $\lambda\geq0$.


Условие дополняющей нежесткости:$\lambda c(x,u)=0$.


Проверим, находится ли неограниченный минимум в области $c(x,u)\leq0$ --
подставим точку из пункта 3.2.1 $(x,u)=(-7/2,19/6)$ в $c(x,u)$:
$$
c\left(-\frac{7}{2},\frac{19}{6}\right)=8\left(-\frac{7}{2}\right)^2+7\left( \frac{19}{6} \right)+2\approx122.17>0
$$
Так как $c(-7/2,19/6)\approx122.17>0$, то неограниченный минимум лежит вне области ограничения --
внутренний минимум невозможен.


Тогда, минимум достигается на границе:
$$
c(x,u)=0,\ \lambda>0,
$$
что соответствует решению из пункта 3.2.2:
$$
(x^*,u^*)=(-0.40167,-0.4701),\ J\approx-9.11394
$$


Таким образом, глобальный минимум достигается на границе $c(x,u)=0$
в точке $(-0.40167,-0.4701)$, значение критерия $J\approx-9.11394$.


\subsection{Градиентный поиск минимума}
\subsubsection{Метод Ньютона-Рафсона}
Оптимизация:
$$
\bar{x}^{(n+1)}=\bar{x}^{(n)}-H^{-1}(\bar{x}^{(n)})\operatorname{grad} J(\bar{x}^{(n)}),
$$
где:
$$
\bar{x}^{(n)}=\begin{bmatrix}
    x(n)\\ u(n)
\end{bmatrix},\ \operatorname{grad} J(\bar{x}^{(n)})=\begin{bmatrix}
    \frac{\partial J(\bar{x}^{(n)})}{\partial x}\\ \frac{\partial J(\bar{x}^{(n)} )}{\partial u}
\end{bmatrix},
$$
$$
H(\bar{x}^{(n)})=\begin{bmatrix}
    \frac{\partial}{\partial x}\left( \frac{\partial J(\bar{x}^{(n)})}{\partial x} \right) &\frac{\partial}{\partial x}\left( \frac{\partial J(\bar{x}^{(n)})}{\partial u} \right)\\
    \frac{\partial}{\partial u}\left( \frac{\partial J(\bar{x}^{(n)})}{\partial x} \right) &\frac{\partial}{\partial u}\left( \frac{\partial J(\bar{x}^{(n)})}{\partial u} \right)
\end{bmatrix}
$$


При нулевой итерации градиент:
$$
\nabla J(x,u)=\begin{bmatrix}
    8x+6u+9\\ 6x+6u+2
\end{bmatrix}
$$
Матрица Гессе и обратная ей:
$$
H=\begin{bmatrix}
    8&6\\ 6 &6
\end{bmatrix},\ H^{-1}=\begin{bmatrix}
    0.5 &-0.5\\ -0.5 &2/3
\end{bmatrix}
$$
Пусть начальная точка $(x_0,u_0)=\begin{bmatrix}
    0&0
\end{bmatrix}^T$, тогда первая итерация:
$$
\bar{x}^{(1)}=\bar{x}^{(0)}-H^{-1}(\bar{x}^{(0)})\operatorname{grad} J(\bar{x}^{(0)})
$$
Вычислим:
$$
\nabla J(\bar{x}^{(0)})=\nabla J(0,0)=\begin{bmatrix}
    9\\2
\end{bmatrix},\ \bar{x}^{(0)}=\begin{bmatrix}
    0\\0
\end{bmatrix},
$$
$$
\bar{x}^{(1)}=-\begin{bmatrix}
    0.5 &-0.5\\ -0.5 &2/3
\end{bmatrix}\begin{bmatrix}
    9\\2
\end{bmatrix}=\begin{bmatrix}
    -7/2\\ 19/6
\end{bmatrix}
$$
Вторая итерация:
$$
\bar{x}^{(2)}=\bar{x}^{(1)}-H^{-1}(\bar{x}^{(1)})\operatorname{grad} J(\bar{x}^{(1)}),
$$
$$
\nabla J(\bar{x}^{(1)})=\nabla J\left( -\frac{7}{2},\frac{19}{6} \right)=\begin{bmatrix}
    0\\0
\end{bmatrix}\Rightarrow -H^{-1}(\bar{x}^{(1)})\nabla J(\bar{x}^{(1)})=\begin{bmatrix}
    0\\0
\end{bmatrix},
$$
Следовательно:
$$
\bar{x}^{(2)}=\bar{x}^{(1)}
$$
Метод Ньютона-Рафсона сходится за одну итерацию -- точка минимума
такая же, как в пункте 3.2.1: $\left( -7/2,19/6 \right)$.


\subsubsection{Метод наискорейшего спуска}
Итерационная формула:
$$
\bar{x}^{(n+1)}=\bar{x}^{(n)}-\gamma\operatorname{grad} J(\bar{x}^{(n)})
$$


Рассмотрим \textbf{колебательную сходимость} при $\gamma=0.12$. Предоставим вычисление итераций матлабу:
\begin{align*}
    &\bar{x}^{(0)}=\begin{bmatrix}
        0 &0
    \end{bmatrix}^T,\ \nabla J(\bar{x}^{(0)})=\begin{bmatrix}
        9&2
    \end{bmatrix}^T,\\
    &\bar{x}^{(1)}=\begin{bmatrix}
        -1.08 &-0.24
    \end{bmatrix}^T,\ \nabla J(\bar{x}^{(1)})=\begin{bmatrix}
        -1.08&-5.92
    \end{bmatrix}^T,\\
    &\bar{x}^{(2)}=\begin{bmatrix}
        -0.9504 &0.4704
    \end{bmatrix}^T,\ \nabla J(\bar{x}^{(2)})=\begin{bmatrix}
        4.2192&-0.88
    \end{bmatrix}^T,\\
    &\bar{x}^{(3)}=\begin{bmatrix}
        -1.4567 &0.576
    \end{bmatrix}^T,\ \nabla J(\bar{x}^{(3)})=\begin{bmatrix}
       0.8024&-3.2842
    \end{bmatrix}^T,\\
    &\bar{x}^{(4)}=\begin{bmatrix}
        -1.553 &0.9701
    \end{bmatrix}^T,\ \nabla J(\bar{x}^{(4)})=\begin{bmatrix}
       2.3967&-1.4973
    \end{bmatrix}^T,\\
    &\bar{x}^{(5)}=\begin{bmatrix}
        -1.8406 &1.1498
    \end{bmatrix}^T,\ \nabla J(\bar{x}^{(5)})=\begin{bmatrix}
       1.1739&-2.1449
    \end{bmatrix}^T,\\
    &\ \ \ \ \ \ \ \ \ \ \ \ \ \ \ \ \ \ \ \ \ \ \ \ \ \ \ \vdots \ \ \ \ \ \ \ \ \ \ \ \ \ \ \ \ \ \ \ \ \ \ \ \ \ \ \ \ \ \ \ \ \ \ \ \ \ \ \ \ \ \ \ \ \ \ \ \ \ \ \ \ \ \ \ \vdots \\
    &\bar{x}^{(8)}=\begin{bmatrix}
        -2.305 &1.7667
    \end{bmatrix}^T,\ \nabla J(\bar{x}^{(8)})=\begin{bmatrix}
       1.1606&-1.2295
    \end{bmatrix}^T,\\
    &\bar{x}^{(9)}=\begin{bmatrix}
        -2.4442 &1.9143
    \end{bmatrix}^T,\ \nabla J(\bar{x}^{(9)})=\begin{bmatrix}
       0.9316&-1.1799
    \end{bmatrix}^T,\\
    &\bar{x}^{(10)}=\begin{bmatrix}
        -2.556 &2.0558
    \end{bmatrix}^T,\ \nabla J(\bar{x}^{(10)})=\begin{bmatrix}
       0.8868&-1.0011
    \end{bmatrix}^T,\\
    &\ \ \ \ \ \ \ \ \ \ \ \ \ \ \ \ \ \ \ \ \ \ \ \ \ \ \ \vdots \ \ \ \ \ \ \ \ \ \ \ \ \ \ \ \ \ \ \ \ \ \ \ \ \ \ \ \ \ \ \ \ \ \ \ \ \ \ \ \ \ \ \ \ \ \ \ \ \ \ \ \ \ \ \ \ \vdots \\
    &\bar{x}^{(19)}=\begin{bmatrix}
        -3.1701 &2.7772
    \end{bmatrix}^T,\ \nabla J(\bar{x}^{(19)})=\begin{bmatrix}
       0.3024&-0.3573
    \end{bmatrix}^T,
\end{align*}
\begin{align*}
    &\bar{x}^{(20)}=\begin{bmatrix}
        -3.2064 &2.8201
    \end{bmatrix}^T,\ \nabla J(\bar{x}^{(20)})=\begin{bmatrix}
       0.2694&-0.3178
    \end{bmatrix}^T,\\
    &\bar{x}^{(21)}=\begin{bmatrix}
        -3.2387 &2.8582
    \end{bmatrix}^T,\ \nabla J(\bar{x}^{(21)})=\begin{bmatrix}
       0.2396&-0.2829
    \end{bmatrix}^T,\\
    &\ \ \ \ \ \ \ \ \ \ \ \ \ \ \ \ \ \ \ \ \ \ \ \ \ \ \ \vdots \ \ \ \ \ \ \ \ \ \ \ \ \ \ \ \ \ \ \ \ \ \ \ \ \ \ \ \ \ \ \ \ \ \ \ \ \ \ \ \ \ \ \ \ \vdots \\
    &\bar{x}^{(171)}\sim\begin{bmatrix}
        -3.5 &3.1667
    \end{bmatrix}^T,\nabla J(\bar{x}^{(171)})\sim\begin{bmatrix}
       0&0
    \end{bmatrix}^T,
\end{align*}
$$
J({x}^{(171)})\sim-19.5833
$$


Получили точку, как в пункте 3.2.1: $(-3.5,3.1667)\sim(-7/2,19/6)$.


Рассмотрим \textbf{апериодическую сходимость} при $\gamma=0.05$:
\begin{align*}
    &\bar{x}^{(0)}=\begin{bmatrix}
        0 &0
    \end{bmatrix}^T,\ \nabla J(\bar{x}^{(0)})=\begin{bmatrix}
        9&2
    \end{bmatrix}^T,\\
    &\bar{x}^{(1)}=\begin{bmatrix}
        -0.45 &-0.1
    \end{bmatrix}^T,\ \nabla J(\bar{x}^{(1)})=\begin{bmatrix}
        4.8&-1.3
    \end{bmatrix}^T,\\
    &\bar{x}^{(2)}=\begin{bmatrix}
        -0.69 &-0.035
    \end{bmatrix}^T,\ \nabla J(\bar{x}^{(2)})=\begin{bmatrix}
        3.27&-2.35
    \end{bmatrix}^T,\\
    &\bar{x}^{(3)}=\begin{bmatrix}
        -0.8535 &0.0825
    \end{bmatrix}^T,\ \nabla J(\bar{x}^{(3)})=\begin{bmatrix}
       2.667&-2.626
    \end{bmatrix}^T,\\
    &\bar{x}^{(4)}=\begin{bmatrix}
        -0.9869 &0.2138
    \end{bmatrix}^T,\ \nabla J(\bar{x}^{(4)})=\begin{bmatrix}
       2.388&-2.6383
    \end{bmatrix}^T,\\
    &\bar{x}^{(5)}=\begin{bmatrix}
        -1.1063 &0.3457
    \end{bmatrix}^T,\ \nabla J(\bar{x}^{(5)})=\begin{bmatrix}
       2.2243&-2.5632
    \end{bmatrix}^T,\\
    &\ \ \ \ \ \ \ \ \ \ \ \ \ \ \ \ \ \ \ \ \ \ \ \ \ \ \ \vdots \ \ \ \ \ \ \ \ \ \ \ \ \ \ \ \ \ \ \ \ \ \ \ \ \ \ \ \ \ \ \ \ \ \ \ \ \ \ \ \ \ \ \ \ \ \ \ \ \ \ \ \ \ \ \ \vdots \\
    &\bar{x}^{(8)}=\begin{bmatrix}
        -1.4227 &0.7147
    \end{bmatrix}^T,\ \nabla J(\bar{x}^{(8)})=\begin{bmatrix}
       1.9066&-2.2481
    \end{bmatrix}^T,\\
    &\bar{x}^{(9)}=\begin{bmatrix}
        -1.518 &0.8271
    \end{bmatrix}^T,\ \nabla J(\bar{x}^{(9)})=\begin{bmatrix}
       1.8184&-2.1456
    \end{bmatrix}^T,\\
    &\bar{x}^{(10)}=\begin{bmatrix}
        -1.6089 &0.9343
    \end{bmatrix}^T,\ \nabla J(\bar{x}^{(10)})=\begin{bmatrix}
       1.7347&-2.0475
    \end{bmatrix}^T,\\
    &\ \ \ \ \ \ \ \ \ \ \ \ \ \ \ \ \ \ \ \ \ \ \ \ \ \ \ \vdots \ \ \ \ \ \ \ \ \ \ \ \ \ \ \ \ \ \ \ \ \ \ \ \ \ \ \ \ \ \ \ \ \ \ \ \ \ \ \ \ \ \ \ \ \ \ \ \ \ \ \ \ \ \ \ \ \vdots \\
    &\bar{x}^{(19)}=\begin{bmatrix}
        -2.2606 &1.7036
    \end{bmatrix}^T,\ \nabla J(\bar{x}^{(19)})=\begin{bmatrix}
       1.1368&-1.342
    \end{bmatrix}^T,\\
    &\bar{x}^{(20)}=\begin{bmatrix}
        -2.3174 &1.7707
    \end{bmatrix}^T,\ \nabla J(\bar{x}^{(20)})=\begin{bmatrix}
       1.0847&-1.2804
    \end{bmatrix}^T,\\
    &\bar{x}^{(21)}=\begin{bmatrix}
        -2.3717 &1.8347
    \end{bmatrix}^T,\ \nabla J(\bar{x}^{(21)})=\begin{bmatrix}
       1.0349&-1.2217
    \end{bmatrix}^T,\\
    &\ \ \ \ \ \ \ \ \ \ \ \ \ \ \ \ \ \ \ \ \ \ \ \ \ \ \ \ \vdots \ \ \ \ \ \ \ \ \ \ \ \ \ \ \ \ \ \ \ \ \ \ \ \ \ \ \ \ \ \ \ \ \ \ \ \ \ \ \ \ \ \ \ \ \ \ \ \ \ \ \ \ \ \ \ \ \ \vdots\\
    &\bar{x}^{(165)}=\begin{bmatrix}
        -3.4987 &3.1651
    \end{bmatrix}^T,\ \nabla J(\bar{x}^{(165)})=\begin{bmatrix}
       0.0012&-0.0014
    \end{bmatrix}^T,
\end{align*}
\begin{align*}
    &\ \ \ \ \ \ \ \ \ \ \ \ \ \ \ \ \ \ \ \ \ \ \ \ \ \ \vdots \ \ \ \ \ \ \ \ \ \ \ \ \ \ \ \ \ \ \ \ \ \ \ \ \ \ \ \ \ \ \ \ \ \ \ \ \ \ \ \ \ \ \ \ \ \ \vdots\\
    &\bar{x}^{(424)}\sim\begin{bmatrix}
        -3.5 &3.1667
    \end{bmatrix}^T,\ \nabla J(\bar{x}^{(424)})\sim\begin{bmatrix}
       0&0
    \end{bmatrix}^T,
\end{align*}
$$
J({x}^{(424)})\sim-19.5833
$$


Получили точку, как в пункте 3.2.1: $(-3.5,3.1667)\sim(-7/2,19/6)$.


\textbf{Замечание}: точность сходимости в обоих случаях составляет $10^{-8}$.
Параметры сходятся несколько раньше при более низкой точности.


\section{Вывод}
В ходе выполнения данной лабораторной работы
были исследованы методы  статической оптимизации:
в случае метода Ньютона-Рафсона сходимости удалось
достичь за 1 итерацию (квадратичная функция),
в случае метода наискорейшего спуска при
колебательной сходимости минимум был достигнут
за 171 итерацию, при апериодической сходимости
за 424 итерации. Точность сходимости последних двух
случаев составляет $10^{-8}$. Также был найден глобальный
минимум критерия качества без ограничений, на границе
ограничения и внутри ограничения. В результате получилось,
что минимум внутри ограничения совпадает с минимумом на
границе ограничения.


\appendix
\renewcommand{\thesection}{\Asbuk{section}}
\section{Приложение}
\begin{lstlisting}[label=code,caption={Программа для расчета итераций}]
%% Criteria and grad
J = @(x, u) 4*x.^2 + 3*u.^2 + 6*x.*u + 9*x + 2*u - 7;
gradJ = @(x, u) [8*x + 6*u + 9; ...
                 6*x + 6*u + 2];

max_iter = 500;
tol = 1e-8;
x0 = 0; u0 = 0;

%% Aperiodic: g = 0.05
gamma_ap = 0.05;

x_hist_ap = zeros(max_iter+1, 1);
u_hist_ap = zeros(max_iter+1, 1);
J_hist_ap = zeros(max_iter+1, 1);
grad_hist_ap = zeros(2, max_iter+1);

x = x0; u = u0;
x_hist_ap(1) = x;
u_hist_ap(1) = u;
J_hist_ap(1) = J(x, u);
grad_hist_ap(:, 1) = gradJ(x, u);

n_ap = 1;

for k = 1:max_iter
    g = grad_hist_ap(:, n_ap);
    
    if norm(g) < tol
        break;
    end
    
    x = x - gamma_ap * g(1);
    u = u - gamma_ap * g(2);
    
    n_ap = n_ap + 1;
    x_hist_ap(n_ap) = x;
    u_hist_ap(n_ap) = u;
    J_hist_ap(n_ap) = J(x, u);
    grad_hist_ap(:, n_ap) = gradJ(x, u);
end

x_hist_ap = x_hist_ap(1:n_ap);
u_hist_ap = u_hist_ap(1:n_ap);
J_hist_ap = J_hist_ap(1:n_ap);
grad_hist_ap = grad_hist_ap(:, 1:n_ap);


%% Oscillating: g = 0.12
gamma_os = 0.12;

x_hist_os = zeros(max_iter+1, 1);
u_hist_os = zeros(max_iter+1, 1);
J_hist_os = zeros(max_iter+1, 1);
grad_hist_os = zeros(2, max_iter+1);

x = x0; u = u0;
x_hist_os(1) = x;
u_hist_os(1) = u;
J_hist_os(1) = J(x, u);
grad_hist_os(:, 1) = gradJ(x, u);

n_os = 1;

for k = 1:max_iter
    g = grad_hist_os(:, n_os);
    
    if norm(g) < tol
        break;
    end
    
    x = x - gamma_os * g(1);
    u = u - gamma_os * g(2);
    
    n_os = n_os + 1;
    x_hist_os(n_os) = x;
    u_hist_os(n_os) = u;
    J_hist_os(n_os) = J(x, u);
    grad_hist_os(:, n_os) = gradJ(x, u);
end

x_hist_os = x_hist_os(1:n_os);
u_hist_os = u_hist_os(1:n_os);
J_hist_os = J_hist_os(1:n_os);
grad_hist_os = grad_hist_os(:, 1:n_os);
\end{lstlisting}
\end{document}