\documentclass[a4paper,14pt]{extarticle}

\usepackage[T2A]{fontenc}
\usepackage[utf8]{inputenc}
\usepackage[english, russian]{babel}

\usepackage[left=30mm, right=10mm, top=20mm, bottom=20mm]{geometry}

\usepackage{tempora}
\usepackage{setspace}
\onehalfspacing

\usepackage{titlesec}
\titleformat{\section}[block]{\bfseries\centering\MakeUppercase}{\thesection.}{1em}{}
\titleformat{\subsection}[block]{\bfseries}{\thesubsection.}{1em}{}
\titleformat{\subsubsection}[block]{\bfseries}{\thesubsubsection.}{1em}{}

\renewcommand{\contentsname}{\hfill \textbf{СОДЕРЖАНИЕ} \hfill\null}

\usepackage{indentfirst}
\setlength{\parindent}{1.25cm}

\usepackage{amsmath, amsfonts, amssymb}
\usepackage{graphicx}
\usepackage{caption}
\usepackage{subcaption}
\usepackage{float}
\usepackage{tikz}
\usetikzlibrary{patterns}
\usepackage{cmap}
\usepackage{hyperref}
\usepackage{xcolor}
\usepackage{listings}

\definecolor{LightGray}{gray}{0.7}

\lstdefinestyle{code}{
    language=matlab, % change if needed
    basicstyle=\small\ttfamily,
    numbers=left,
    numberstyle=\small\color{LightGray},
    stepnumber=1,
    numbersep=5pt,
    backgroundcolor=\color{white},
    showspaces=false,
    showstringspaces=false,
    showtabs=false,
    tabsize=4,
    captionpos=b,
    breaklines=true,
    breakatwhitespace=false,
    frame=single,
    rulecolor=\color{LightGray},
    linewidth=\linewidth,
    keywordstyle=\color{blue}\bfseries,
    commentstyle=\color{green!40!black},
    stringstyle=\color{violet},
    escapeinside={\%*}{*)},
    xleftmargin=10pt,
    xrightmargin=10pt,
    framexleftmargin=0pt,
    framexrightmargin=0pt
}
\lstset{style=code}

\hypersetup{
    colorlinks=true,
    linkcolor=blue,
    filecolor=magenta,
    urlcolor=cyan,
    pdftitle={lab arc},
    pdfauthor={Rumyantsev Alexey},
    pdfsubject={control},
    pdfkeywords={LaTeX, PDF},
    pdfpagemode=FullScreen,
}

\graphicspath{{src/images/}}

\begin{document}

\begin{titlepage}
    \begin{center}
        МИНИСТЕРСТВО НАУКИ И ВЫСШЕГО ОБРАЗОВАНИЯ РОССИЙСКОЙ ФЕДЕРАЦИИ\\
        \vspace*{2.5mm}
        Федеральное государственное автономное образовательное учреждение высшего образования
        «НАЦИОНАЛЬНЫЙ ИССЛЕДОВАТЕЛЬСКИЙ УНИВЕРСИТЕТ ИТМО»\\
        \vspace*{2.5mm}
        ФАКУЛЬТЕТ СИСТЕМ УПРАВЛЕНИЯ И РОБОТОТЕХНИКИ
        \vfill

        {\large\bfseries ОТЧЕТ ПО ЛАБОРАТОРНОЙ РАБОТЕ №1}\\
        {\large по дисциплине}\\
        {\large«ТЕОРИЯ ОПТИМАЛЬНОГО УПРАВЛЕНИЯ»}\\
        {\large на тему}\\
        {\large «ПОИСК МИНИМУМА С ПОМОЩЬЮ МЕТОДОВ СТАТИЧЕСКОЙ ОПТИМИЗАЦИИ»}\\
        Вариант 31
        \vfill

        \begin{flushright}
            Выполнил: студент гр. R3441\\
            Румянцев А. А.\medskip\\

            Проверил: преподаватель\\
            Парамонов А. В.
        \end{flushright}
        \vfill

        Санкт-Петербург\\
        2025
    \end{center}
\end{titlepage}

\setcounter{page}{2}
\tableofcontents
\newpage


\section{Цель работы}
Исследовать методы статической оптимизации
и найти минимум критерия качества $J(x,u)$.


\section{Постановка задачи}
Дан критерий качества $J(x,u)$ и ограничение $c(x,u)$.
Необходимо:
\begin{enumerate}
    \item Найти глобальный минимум $J(x,u)$ на основе необходимого и достаточного условий экстремума:
    \begin{enumerate}
        \item Без ограничений;
        \item С ограничением в виде равенства $c(x,u)=0$;
        \item С ограничением в виде неравенства $c(x,u)\leq0$.
    \end{enumerate}
    \item Осуществить градиентный поиск минимума критерия качества $J_1(x,u)=J(x,u)$:
    \begin{enumerate}
        \item Методом Ньютона Рафсона произвести пошаговый расчет экстремума.
        \item Методом наискорейшего спуска для двух различных $\gamma$
        (соответствующих колебательной и апериодической сходимостям) произвести
        пошаговый расчет экстремума.
    \end{enumerate}    
\end{enumerate}


\section{Экспериментальная часть}
\subsection{Исходные данные}
Согласно варианту 31, критерий качества:
$$
J(x,u)=4x^2+3u^2+6xu+9x+2u-7
$$
Ограничение:
$$
c(x,u)=8x^2+7u+2
$$


\subsection{Поиск глобального минимума}
\subsubsection{Без ограничений}
Градиент критерия:
$$
\operatorname{grad}{J(x,u)}=\nabla J(x,u)=\begin{bmatrix}
    \frac{\partial J(x,u)}{\partial x}\\ \frac{\partial J(x,u)}{\partial u}
\end{bmatrix}=\begin{bmatrix}
    8x+6u+9\\ 6x+6u+2
\end{bmatrix}
$$
Приравняем производные к нулю и найдем $x,u$:
$$
\begin{cases}
    8x+6u+9=0,\\ 6x+6u+2=0
\end{cases}\Rightarrow \begin{cases}
    x=-7/2,\\u=19/6
\end{cases}
$$
Вычислим значение критерия:
$$
J\left(-\frac{7}{2},\frac{19}{6}\right)=4\left( -\frac{7}{2} \right)^2+
3\left( \frac{19}{6} \right)^2+
6\left( -\frac{7}{2} \right)\left( \frac{19}{6} \right)+
9\left( -\frac{7}{2} \right)+2\left( \frac{19}{6} \right)-7,
$$
$$
J\left(-\frac{7}{2},\frac{19}{6}\right)\approx-19.583
$$
Таким образом, для точки $\left(-7/2,19/6\right)$
значение критерия $J\approx-19.583$ -- глобальный минимум без ограничений.


\subsubsection{С ограничением в виде равенства}
Ограничение: $c(x,u)=8x^2+7u+2=0$.


Функция Лагранжа:
$$
L(x,u,\lambda)=J(x,u)+\lambda c(x,u),
$$
где $\lambda$ -- множитель Лагранжа.


Подставим:
$$
L(x,u,\lambda)=4x^2+3u^2+6xu+9x+2u-7+\lambda\left( 8x^2+7u+2 \right)
$$
Система частных производных лагранжиана:
$$
\begin{cases}
    \frac{\partial L}{\partial \lambda}=c(x,u)=8x^2+7u+2=0,\\
    \frac{\partial L}{\partial x}=8x+6u+9+16x\lambda=0,\\
    \frac{\partial L}{u}=6x+6u+2+7\lambda=0
\end{cases}
$$
Выразим $\lambda$:
$$
\lambda=-\frac{6x+6u+2}{7}
$$
Подставим в $\partial L/\partial x$:
$$
8x+6u+9-16x\left( \frac{6x+6u+2}{7} \right)=0,
$$
$$
7\left( 8x+6u+9 \right)-16x\left( 6x+6u+2 \right)=0,
$$
$$
-96x^2-96xu+24x+42u+63=0
$$
Выразим $u$ из $\partial L/\partial \lambda$:
$$
u=-\frac{8x^2+2}{7}
$$
Подставим:
$$
-96x^2+96x\left( \frac{8x^2+2}{7} \right)+24x-42\left( \frac{8x^2+2}{7} \right)+63=0,
$$
$$
256x^3-336x^2+120x+119=0\Rightarrow\begin{cases}
    x_1\approx-0.40167,\\
    x_{2,3}\approx0.85708\pm0.65014i
\end{cases}
$$
В реальных задачах комплексные решения лагранжевых уравнений отбрасываются.


Подставим $x_1=-0.40167$ в $u$:
$$
u=-\frac{8\left( -0.40167 \right)^2+2}{7}\approx-0.4701
$$
Вычислим $J$ при $x=-0.40167,u=-0.4701$:
\begin{align*}
    J(-0.40167,-0.4701)= &4(-0.40167)^2+3(-0.4701)^2+\\
    &+6(-0.40167)(-0.4701)+9(-0.40167)+\\
    &+2(-0.4701)-7=\approx -9.11394
\end{align*}


Так как реальное решение единственное,
то единственным кандидатом на экстремум на границе $c(x,u)=0$ является точка $(-0.40167,-0.4701)$,
значение критерия в которой $J\approx-9.11394$ -- глобальный минимум с ограничением в виде равенства.


\subsubsection{С ограничением в виде неравенства}
Ограничение: $c(x,u)=8x^2+7u+2\leq0$.


Условие Куна-Таккера: $\lambda\geq0$.


Условие дополняющей нежесткости:$\lambda c(x,u)=0$.


Проверим, находится ли неограниченный минимум в области $c(x,u)\leq0$ --
подставим точку из пункта 3.2.1 $(x,u)=(-7/2,19/6)$ в $c(x,u)$:
$$
c\left(-\frac{7}{2},\frac{19}{6}\right)=8\left(-\frac{7}{2}\right)^2+7\left( \frac{19}{6} \right)+2\approx122.17>0
$$
Так как $c(-7/2,19/6)\approx122.17>0$, то неограниченный минимум лежит вне области ограничения --
внутренний минимум невозможен.


Тогда, минимум достигается на границе:
$$
c(x,u)=0,\ \lambda>0,
$$
что соответствует решению из пункта 3.2.2:
$$
(x^*,u^*)=(-0.40167,-0.4701),\ J\approx-9.11394
$$


Таким образом, глобальный минимум достигается на границе $c(x,u)=0$
в точке $(-0.40167,-0.4701)$, значение критерия $J\approx-9.11394$.
\end{document}