\documentclass[a4paper,14pt]{extarticle}

\usepackage[T2A]{fontenc}
\usepackage[utf8]{inputenc}
\usepackage[english, russian]{babel}

\usepackage[left=30mm, right=10mm, top=20mm, bottom=20mm]{geometry}

\usepackage{tempora}
\usepackage{setspace}
\onehalfspacing

\usepackage{titlesec}
\titleformat{\section}[block]{\bfseries\centering\MakeUppercase}{\thesection.}{1em}{}
\titleformat{\subsection}[block]{\bfseries}{\thesubsection.}{1em}{}
\titleformat{\subsubsection}[block]{\bfseries}{\thesubsubsection.}{1em}{}

\renewcommand{\contentsname}{\hfill \textbf{СОДЕРЖАНИЕ} \hfill\null}

\usepackage{indentfirst}
\setlength{\parindent}{1.25cm}

\usepackage{amsmath, amsfonts, amssymb}
\usepackage{graphicx}
\usepackage{caption}
\usepackage{subcaption}
\usepackage{float}
\usepackage{tikz}
\usetikzlibrary{patterns}
\usepackage{cmap}
\usepackage{hyperref}
\usepackage{xcolor}
\usepackage{listings}

\definecolor{LightGray}{gray}{0.7}

\lstdefinestyle{code}{
    language=matlab, % change if needed
    basicstyle=\small\ttfamily,
    numbers=left,
    numberstyle=\small\color{LightGray},
    stepnumber=1,
    numbersep=5pt,
    backgroundcolor=\color{white},
    showspaces=false,
    showstringspaces=false,
    showtabs=false,
    tabsize=4,
    captionpos=b,
    breaklines=true,
    breakatwhitespace=false,
    frame=single,
    rulecolor=\color{LightGray},
    linewidth=\linewidth,
    keywordstyle=\color{blue}\bfseries,
    commentstyle=\color{green!40!black},
    stringstyle=\color{violet},
    escapeinside={\%*}{*)},
    xleftmargin=10pt,
    xrightmargin=10pt,
    framexleftmargin=0pt,
    framexrightmargin=0pt
}
\lstset{style=code}

\hypersetup{
    colorlinks=true,
    linkcolor=blue,
    filecolor=magenta,
    urlcolor=cyan,
    pdftitle={lab arc},
    pdfauthor={Rumyantsev Alexey},
    pdfsubject={control},
    pdfkeywords={LaTeX, PDF},
    pdfpagemode=FullScreen,
}

\graphicspath{{src/images/}}

\begin{document}

\begin{titlepage}
    \begin{center}
        МИНИСТЕРСТВО НАУКИ И ВЫСШЕГО ОБРАЗОВАНИЯ РОССИЙСКОЙ ФЕДЕРАЦИИ\\
        \vspace*{2.5mm}
        Федеральное государственное автономное образовательное учреждение высшего образования
        «НАЦИОНАЛЬНЫЙ ИССЛЕДОВАТЕЛЬСКИЙ УНИВЕРСИТЕТ ИТМО»\\
        \vspace*{2.5mm}
        ФАКУЛЬТЕТ СИСТЕМ УПРАВЛЕНИЯ И РОБОТОТЕХНИКИ
        \vfill

        {\large\bfseries ОТЧЕТ ПО ЛАБОРАТОРНОЙ РАБОТЕ №6}\\
        {\large по дисциплине}\\
        {\large«ТЕОРИЯ ОПТИМАЛЬНОГО УПРАВЛЕНИЯ»}\\
        {\large на тему}\\
        {\large «СИНТЕЗ ОПТИМАЛЬНОГО УПРАВЛЕНИЯ. Н$_\infty$-ОПТИМИЗАЦИЯ»}\\
        Вариант 31
        \vfill

        \begin{flushright}
            Выполнил: студент гр. R3441\\
            Румянцев А. А.\medskip\\

            Проверил: преподаватель\\
            Парамонов А. В.
        \end{flushright}
        \vfill

        Санкт-Петербург\\
        2025
    \end{center}
\end{titlepage}

\setcounter{page}{2}
\tableofcontents
\newpage


\section{Цель работы}
Исследовать $H_{\infty}$-оптимальный регулятор и определить
$H_{\infty}$-нормы передаточных функций.


\section{Постановка задачи}
Дан возмущённый линейный объект управления:
$$
\dot{x}=Ax+Bu+B_ff,\ x(0)=\begin{bmatrix}
    1\\0
\end{bmatrix}
$$
Возмущение:
$$
f=10\sin{6t}+5\cos{2t}+4\cos{3t}+3\cos{8t}
$$
Необходимо:
\begin{enumerate}
    \item Построить $H_\infty$-оптимальный регулятор вида
    $u=Kx$. Расчет произвести на
    основе уравнения Риккати:
    $$
    \begin{cases}
        A^TP+PA+Q-PBB^TP+\gamma^{-2}PB_fB_f^TP=0,\\
        K=-B^TP
    \end{cases}
    $$
    \item Экспериментально определить минимальное значение коэффициента
    $\gamma=\gamma_{\min}$, при котором существует положительно полуопределённая
    матрица $P$ в качестве решения уравнения Риккати.
    \item Построить графики управления $u$ и переменных состояния $x_1,x_2$ для $\gamma_{\min}$.
    \item Определить $H_\infty$-нормы передаточных функций $C_1\left( Is-\left( A+BK \right) \right)^{-1}B_f$
    и $C_2\left( Is-\left( A+BK \right) \right)^{-1}B_f$, где $C_1=\left[ 1,0 \right]$ и $C_2=[0,1]$.
    \item Определить $H_\infty$-норму передаточной функции $\left( Is-\left( A+BK \right) \right)^{-1}B_f$.
\end{enumerate}


\section{Экспериментальная часть}
\subsection{Исходные данные}
Согласно варианту 31, матрицы $A,B,B_f,Q$:
$$
A=\begin{bmatrix}
    7&-4\\ 5&6
\end{bmatrix},\
B=\begin{bmatrix}
    5\\2
\end{bmatrix},\
B_f=\begin{bmatrix}
    3\\9
\end{bmatrix},\
Q=\begin{bmatrix}
    3&0\\ 0&4
\end{bmatrix}
$$


\subsection{Расчет оптимального регулятора}
Выберем $\gamma=10$ и проведем расчет $K$:
$$
K=\begin{bmatrix}
    -1.6711&  -15.2927
\end{bmatrix}
$$
Собственные числа замкнутой системы:
$$
\sigma\left( A+BK \right)=\left\{ -11.7411,-14.1997 \right\}
$$
Замкнутая система асимптотически устойчива.


\subsection{Минимальное значение коэффициента}
Экспериментально подобран минимальный коэффициент
$\gamma_{\min}=5.6061$, при котором существует положительно полуопределённая матрица
$P$ как решение уравнения Риккати:
$$
P=\begin{bmatrix}
    2530&   -28630\\
   -28630&    323900
\end{bmatrix},\ \sigma(P)=\left\{ 0,326430 \right\}
$$
$$
K=\begin{bmatrix}
    44600   -504670
\end{bmatrix},\ \sigma\left( A+BK \right)=\left\{ -10,-786330 \right\}
$$


\subsection{Моделирование системы}
Схема моделирования:
\begin{figure}[H]
    \centering
    \includegraphics[scale=0.6]{sch.png}
    \caption{Схема моделирования замкнутой системы}
    \label{fig:sch}
\end{figure}


Результаты моделирования:
\begin{figure}[H]
    \centering
    \includegraphics[scale=1]{uu.png}
    \caption{Управление $u(t)$ для $\gamma_{\min}$ полностью}
    \label{fig:uu}
\end{figure}
\begin{figure}[H]
    \centering
    \includegraphics[scale=1]{uuu.png}
    \caption{Управление $u(t)$ для $\gamma_{\min}$ вблизи (1)}
    \label{fig:uuu}
\end{figure}
\begin{figure}[H]
    \centering
    \includegraphics[scale=1]{u.png}
    \caption{Управление $u(t)$ для $\gamma_{\min}$ вблизи (2)}
    \label{fig:u}
\end{figure}
\begin{figure}[H]
    \centering
    \includegraphics[scale=1]{x.png}
    \caption{Состояния системы $x(t)$ для $\gamma_{\min}$}
    \label{fig:x}
\end{figure}


На графике управления заметно значительное перерегулироване в начале.


\subsection{Вычисление $\boldsymbol{H_{\infty}}$-нормы для передаточных функций $\boldsymbol{W_1,W_2}$}
Определим $H_\infty$-нормы передаточных функций $W_1,W_2,C_1=\left[ 1,0 \right],C_2=[0,1]$:
$$
W_1=C_1\left( Is-\left( A+BK \right) \right)^{-1}B_f,\ ||W_1||_\infty=2.7162,\\
$$
$$
W_2=C_2\left( Is-\left( A+BK \right) \right)^{-1}B_f,\ ||W_2||_\infty=0.24
$$
Возмущение через $B_f$ влияет на первую координату состояния в $11.3175$ раз сильнее, чем на вторую.


\subsection{Вычисление $\boldsymbol{H_{\infty}}$-нормы для передаточной функции $\boldsymbol{W}$}
Определим $H_\infty$-норму передаточной функции $W,C=I$:
$$
W=\left( Is-\left( A+BK \right) \right)^{-1}B_f,\ ||W||_\infty=2.7268
$$
Так как $||W||_\infty<\gamma=5.6061$,
замкнутая система обеспечивает
гарантированное подавление возмущений:
при любом входном воздействии
конечной энергии энергия
выходного сигнала не превышает
$\gamma^2$-кратной энергии возмущения.


\section{Вывод}
В ходе выполнения лабораторной
работы был синтезирован $H_\infty$-оптимальный
регулятор для линейного объекта с возмущениями.
Также были исследованы $H_\infty$-нормы передаточных
функций -- так как они меньше выбранного $\gamma$,
то у системы есть запас робастности.


\appendix
\renewcommand{\thesection}{\Asbuk{section}}
\section{Приложение}
\begin{lstlisting}[label=code,caption={Программа для лабораторной работы}]
%% plant parameters
A=[7 -4;
    5 6];
B=[5;
    2];
Bf=[3;
    9];
Q=[3 0;
    0 4];
%% solve Riccati
g=5.6061;
P=are(A,B*B'-g^(-2)*Bf*Bf',Q)
K=-B'*P
eP=eig(P)
eK=eig(A+B*K)
%% H_infty-norm
C1=[1 0];
C2=[0 1];
Acl=A+B*K;

sys1ss=ss(Acl,Bf,C1,0);
[ninf1,fpeak1]=hinfnorm(sys1ss)
sys2ss=ss(Acl,Bf,C2,0);
[ninf2,fpeak2]=hinfnorm(sys2ss)
sys3ss=ss(Acl,Bf,eye(2),0);
[ninf3,fpeak3]=hinfnorm(sys3ss)
\end{lstlisting}
\end{document}