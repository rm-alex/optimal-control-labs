\documentclass[a4paper,14pt]{extarticle}

\usepackage[T2A]{fontenc}
\usepackage[utf8]{inputenc}
\usepackage[english, russian]{babel}

\usepackage[left=30mm, right=10mm, top=20mm, bottom=20mm]{geometry}

\usepackage{tempora}
\usepackage{setspace}
\onehalfspacing

\usepackage{titlesec}
\titleformat{\section}[block]{\bfseries\centering\MakeUppercase}{\thesection.}{1em}{}
\titleformat{\subsection}[block]{\bfseries}{\thesubsection.}{1em}{}
\titleformat{\subsubsection}[block]{\bfseries}{\thesubsubsection.}{1em}{}

\renewcommand{\contentsname}{\hfill \textbf{СОДЕРЖАНИЕ} \hfill\null}

\usepackage{indentfirst}
\setlength{\parindent}{1.25cm}

\usepackage{amsmath, amsfonts, amssymb}
\usepackage{graphicx}
\usepackage{caption}
\usepackage{subcaption}
\usepackage{float}
\usepackage{tikz}
\usetikzlibrary{patterns}
\usepackage{cmap}
\usepackage{hyperref}
\usepackage{xcolor}
\usepackage{listings}

\definecolor{LightGray}{gray}{0.7}

\lstdefinestyle{code}{
    language=matlab, % change if needed
    basicstyle=\small\ttfamily,
    numbers=left,
    numberstyle=\small\color{LightGray},
    stepnumber=1,
    numbersep=5pt,
    backgroundcolor=\color{white},
    showspaces=false,
    showstringspaces=false,
    showtabs=false,
    tabsize=4,
    captionpos=b,
    breaklines=true,
    breakatwhitespace=false,
    frame=single,
    rulecolor=\color{LightGray},
    linewidth=\linewidth,
    keywordstyle=\color{blue}\bfseries,
    commentstyle=\color{green!40!black},
    stringstyle=\color{violet},
    escapeinside={\%*}{*)},
    xleftmargin=10pt,
    xrightmargin=10pt,
    framexleftmargin=0pt,
    framexrightmargin=0pt
}
\lstset{style=code}

\hypersetup{
    colorlinks=true,
    linkcolor=blue,
    filecolor=magenta,
    urlcolor=cyan,
    pdftitle={lab arc},
    pdfauthor={Rumyantsev Alexey},
    pdfsubject={control},
    pdfkeywords={LaTeX, PDF},
    pdfpagemode=FullScreen,
}

\graphicspath{{src/images/}}

\begin{document}

\begin{titlepage}
    \begin{center}
        МИНИСТЕРСТВО НАУКИ И ВЫСШЕГО ОБРАЗОВАНИЯ РОССИЙСКОЙ ФЕДЕРАЦИИ\\
        \vspace*{2.5mm}
        Федеральное государственное автономное образовательное учреждение высшего образования
        «НАЦИОНАЛЬНЫЙ ИССЛЕДОВАТЕЛЬСКИЙ УНИВЕРСИТЕТ ИТМО»\\
        \vspace*{2.5mm}
        ФАКУЛЬТЕТ СИСТЕМ УПРАВЛЕНИЯ И РОБОТОТЕХНИКИ
        \vfill

        {\large\bfseries ОТЧЕТ ПО ЛАБОРАТОРНОЙ РАБОТЕ №3}\\
        {\large по дисциплине}\\
        {\large«ТЕОРИЯ ОПТИМАЛЬНОГО УПРАВЛЕНИЯ»}\\
        {\large на тему}\\
        {\large «СИНТЕЗ ОПТИМАЛЬНОГО РЕГУЛЯТОРА ДЛЯ ЛИНЕЙНОГО СТАЦИОНАРНОГО ОБЪЕКТА»}\\
        Вариант 31
        \vfill

        \begin{flushright}
            Выполнил: студент гр. R3441\\
            Румянцев А. А.\medskip\\

            Проверил: преподаватель\\
            Парамонов А. В.
        \end{flushright}
        \vfill

        Санкт-Петербург\\
        2025
    \end{center}
\end{titlepage}

\setcounter{page}{2}
\tableofcontents
\newpage

\section{Цель работы}
Разработать и исследовать оптимальный регулятор
для линейного объекта управления на основе решения
уравнения Риккати, оценить его эффективность по критерию
качества, а также проанализировать влияние
параметров регулятора
на динамику системы и
значение целевого функционала.


\section{Постановка задачи}
Дан линейный объект:
$$
\dot{x}=Ax+bu,\ x(0)
$$


Необходимо рассчитать коэффициенты оптимального регулятора
для этого объекта.


\section{Теоретическая часть}
Структура регулятора $u=-Kx$.
Расчет произвести на основе уравнения Риккати
$$
\begin{cases}
    A^TP+PA+Q-Pbr^{-1}b^TP=0,\\
    K=r^{-1}b^TP,
\end{cases}
$$
и критерия качества вида
$$
J=\int\limits_{0}^{\infty}x^T(\tau)Qx(\tau)+ru^2(\tau)\,d\tau
$$


\section{Экспериментальная часть}
\subsection{Исходные данные}
Согласно варианту 31, матрицы $A,b,Q$:
$$
A=\begin{bmatrix}
    0&1\\ -2&4
\end{bmatrix},\
b=\begin{bmatrix}
    8\\ 9
\end{bmatrix},\
Q=\begin{bmatrix}
    3&0\\ 0&4
\end{bmatrix}
$$
Параметр $r=4$.


\subsection{Коэффициенты оптимального регулятора}
Программа для подсчета $K$:
\begin{lstlisting}[label=code1,caption={Программа MATLAB для вычисления K через Риккати}]
%% plant parameters
A=[0 1;
    -2 4];
b=[8;
    9];
Q=[3 0;
    0 4];
r=4;
v = 1;

%% solve Riccati
[P,K,e]=icare(A,sqrt(v)*b,Q,r);
P
K=inv(r)*b'*P
eK=eig(A-b*K)
\end{lstlisting}


Результаты:
$$
P\approx\begin{bmatrix}
    40.795  &-39.067\\
  -39.067   &38.101
\end{bmatrix},\ K=\begin{bmatrix}
    -6.310    &7.594
\end{bmatrix}
$$


Спектр замкнутой системы:
$$
\sigma\left( A-BK \right)=\left\{ -2.1961, -11.6695 \right\}
$$
Замкнутая система асимптотически устойчива.


\subsection{Моделирование замкнутой системы}
Проведем моделирование замкнутой системы при начальных условиях:
$$
x(0)=\begin{bmatrix}
    1\\0
\end{bmatrix}
$$


Схема моделирования:
\begin{figure}[H]
    \centering
    \includegraphics[scale=0.6]{sch.png}
    \caption{Схема моделирования замкнутой системы, $u=-Kx$}
    \label{fig:sch}
\end{figure}


Результаты моделирования:
\begin{figure}[H]
    \centering
    \includegraphics[scale=1]{1x.png}
    \caption{Вектор состояния объекта $x(t)$, $u=-Kx$}
    \label{fig:1x}
\end{figure}
\begin{figure}[H]
    \centering
    \includegraphics[scale=1]{1u.png}
    \caption{Управление $u=-Kx$}
    \label{fig:1u}
\end{figure}
\begin{figure}[H]
    \centering
    \includegraphics[scale=1]{1J.png}
    \caption{Критерий качества $J(t)$, $u=-Kx$}
    \label{fig:1J}
\end{figure}
Установившееся значение критерия качества на
интервале моделирования составило $J=40.795$.


\subsection{Коэффициенты оптимального регулятора с отклонением}
Отклоним расчетные значения $K$ на $5\%$:
$$
K_{b}=1.05K=\begin{bmatrix}
    -6.6255    &7.9737
\end{bmatrix}
$$


Спектр замкнутой системы:
$$
\sigma\left( A-BK_b \right)=\left\{ -2.1211, -12.6382 \right\}
$$
Замкнутая система осталась асимптотически устойчивой.


\subsection{Моделирование системы с регулятором с отклонением}
Проведем аналогичное моделирование:
\begin{figure}[H]
    \centering
    \includegraphics[scale=1]{2x.png}
    \caption{Вектор состояния объекта $x(t)$, $u=-K_bx$}
    \label{fig:2x}
\end{figure}
\begin{figure}[H]
    \centering
    \includegraphics[scale=1]{2u.png}
    \caption{Управление $u=-K_bx$}
    \label{fig:2u}
\end{figure}
\begin{figure}[H]
    \centering
    \includegraphics[scale=1]{2J.png}
    \caption{Критерий качества $J(t)$, $u=-K_bx$}
    \label{fig:2J}
\end{figure}
Установившееся значение критерия качества на
интервале моделирования составило $J=40.810$.


Небольшое отклонение коэффициентов оптимального
регулятора\\ несколько увеличило установившееся
значение критерия качества $J$: $40.810>40.795$,
то есть штрафы за отклонение состояния от нуля $Q$
и на затраты на управление $r$ увеличились,
регулятор стал менее эффективным.


\subsection{Исследование параметров $\boldsymbol{r},\boldsymbol{Q}$}
Проведем моделирование для параметров $r\in\left\{ 1,4,10 \right\}$
и $k\in\left\{ 0.5,1,2 \right\}:Q_i=k_iQ^*$, $Q^*$ -- исходная матрица $Q$.


Результаты моделирования:
\begin{figure}[H]
    \centering
    \includegraphics[scale=1]{3xQ1.png}
    \caption{Вектор состояния объекта $x(t)$, $Q_{j=1},r_i$}
    \label{fig:3xQ1}
\end{figure}
\begin{figure}[H]
    \centering
    \includegraphics[scale=1]{3xQ2.png}
    \caption{Вектор состояния объекта $x(t)$, $Q_{j=2},r_i$}
    \label{fig:3xQ2}
\end{figure}
\begin{figure}[H]
    \centering
    \includegraphics[scale=1]{3xQ3.png}
    \caption{Вектор состояния объекта $x(t)$, $Q_{j=3},r_i$}
    \label{fig:3xQ3}
\end{figure}


\begin{figure}[H]
    \centering
    \includegraphics[scale=1]{3uQ1.png}
    \caption{Управление $u=-K_{j,i}x$, $Q_{j=1},r_i$}
    \label{fig:3uQ1}
\end{figure}
\begin{figure}[H]
    \centering
    \includegraphics[scale=1]{3uQ2.png}
    \caption{Управление $u=-K_{j,i}x$, $Q_{j=2},r_i$}
    \label{fig:3uQ2}
\end{figure}
\begin{figure}[H]
    \centering
    \includegraphics[scale=1]{3uQ3.png}
    \caption{Управление $u=-K_{j,i}x$, $Q_{j=3},r_i$}
    \label{fig:3uQ3}
\end{figure}


\begin{figure}[H]
    \centering
    \includegraphics[scale=1]{3JQ1.png}
    \caption{Критерий качества $J(t)$, $Q_{j=1},r_i$}
    \label{fig:3JQ1}
\end{figure}
\begin{figure}[H]
    \centering
    \includegraphics[scale=1]{3JQ2.png}
    \caption{Критерий качества $J(t)$, $Q_{j=2},r_i$}
    \label{fig:3JQ2}
\end{figure}
\begin{figure}[H]
    \centering
    \includegraphics[scale=1]{3JQ3.png}
    \caption{Критерий качества $J(t)$, $Q_{j=3},r_i$}
    \label{fig:3JQ3}
\end{figure}


Чем меньше штраф на затраты на управление, тем быстрее
система сходится к нулю, однако управления затрачивается
больше.


Меньше управления затрачивается в обратном случае,
когда штраф на затраты на управление больше.


Чем больше штраф на скорость сходимости, тем быстрее
система приходит к устоявшемуся состоянию. Однако
скорость требует б\'{о}льших затрат на управление.


Исходя из графиков, минимальный и максимальный устоявшиеся критерии
$J_{\min}\sim18,J_{\max}\sim84$. Таким образом, оптимальный
по времени сходимости и затратам на управление устоявшийся критерий должен быть $J_\text{opt}\sim51$.
Самый близкий из рассматриваемых случаев -- при $k=1 (Q_{j=2}),r=10$ (см. рис. \ref{fig:3JQ2}).


\section{Вывод}
В ходе выполнения лабораторной работы
был синтезирован оптимальный регулятор.
Результаты моделирования показали корректность
проведенных расчетов. Небольшое отклонение
коэффициентов регулятора приводит к некоторому
увеличению критерия качества. Исследование
при различных $Q,r$ позволило выбрать
наиболее оптимальный регулятор для данной системы
в рамках исследуемых $k,r$.
\end{document}