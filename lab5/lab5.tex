\documentclass[a4paper,14pt]{extarticle}

\usepackage[T2A]{fontenc}
\usepackage[utf8]{inputenc}
\usepackage[english, russian]{babel}

\usepackage[left=30mm, right=10mm, top=20mm, bottom=20mm]{geometry}

\usepackage{tempora}
\usepackage{setspace}
\onehalfspacing

\usepackage{titlesec}
\titleformat{\section}[block]{\bfseries\centering\MakeUppercase}{\thesection.}{1em}{}
\titleformat{\subsection}[block]{\bfseries}{\thesubsection.}{1em}{}
\titleformat{\subsubsection}[block]{\bfseries}{\thesubsubsection.}{1em}{}

\renewcommand{\contentsname}{\hfill \textbf{СОДЕРЖАНИЕ} \hfill\null}

\usepackage{indentfirst}
\setlength{\parindent}{1.25cm}

\usepackage{amsmath, amsfonts, amssymb}
\usepackage{graphicx}
\usepackage{caption}
\usepackage{subcaption}
\usepackage{float}
\usepackage{tikz}
\usetikzlibrary{patterns}
\usepackage{cmap}
\usepackage{hyperref}
\usepackage{xcolor}
\usepackage{listings}

\definecolor{LightGray}{gray}{0.7}

\lstdefinestyle{code}{
    language=matlab, % change if needed
    basicstyle=\small\ttfamily,
    numbers=left,
    numberstyle=\small\color{LightGray},
    stepnumber=1,
    numbersep=5pt,
    backgroundcolor=\color{white},
    showspaces=false,
    showstringspaces=false,
    showtabs=false,
    tabsize=4,
    captionpos=b,
    breaklines=true,
    breakatwhitespace=false,
    frame=single,
    rulecolor=\color{LightGray},
    linewidth=\linewidth,
    keywordstyle=\color{blue}\bfseries,
    commentstyle=\color{green!40!black},
    stringstyle=\color{violet},
    escapeinside={\%*}{*)},
    xleftmargin=10pt,
    xrightmargin=10pt,
    framexleftmargin=0pt,
    framexrightmargin=0pt
}
\lstset{style=code}

\hypersetup{
    colorlinks=true,
    linkcolor=blue,
    filecolor=magenta,
    urlcolor=cyan,
    pdftitle={lab arc},
    pdfauthor={Rumyantsev Alexey},
    pdfsubject={control},
    pdfkeywords={LaTeX, PDF},
    pdfpagemode=FullScreen,
}

\graphicspath{{src/images/}}

\begin{document}

\begin{titlepage}
    \begin{center}
        МИНИСТЕРСТВО НАУКИ И ВЫСШЕГО ОБРАЗОВАНИЯ РОССИЙСКОЙ ФЕДЕРАЦИИ\\
        \vspace*{2.5mm}
        Федеральное государственное автономное образовательное учреждение высшего образования
        «НАЦИОНАЛЬНЫЙ ИССЛЕДОВАТЕЛЬСКИЙ УНИВЕРСИТЕТ ИТМО»\\
        \vspace*{2.5mm}
        ФАКУЛЬТЕТ СИСТЕМ УПРАВЛЕНИЯ И РОБОТОТЕХНИКИ
        \vfill

        {\large\bfseries ОТЧЕТ ПО ЛАБОРАТОРНОЙ РАБОТЕ №5}\\
        {\large по дисциплине}\\
        {\large«ТЕОРИЯ ОПТИМАЛЬНОГО УПРАВЛЕНИЯ»}\\
        {\large на тему}\\
        {\large «СИНТЕЗ ОПТИМАЛЬНОГО НАБЛЮДАТЕЛЯ (ФИЛЬТРА КАЛМАНА) И ЛКГ-СИНТЕЗ»}\\
        Вариант 31
        \vfill

        \begin{flushright}
            Выполнил: студент гр. R3441\\
            Румянцев А. А.\medskip\\

            Проверил: преподаватель\\
            Парамонов А. В.
        \end{flushright}
        \vfill

        Санкт-Петербург\\
        2025
    \end{center}
\end{titlepage}

\setcounter{page}{2}
\tableofcontents
\newpage


\section{Цель работы}
Исследовать оптимальный наблюдатель (фильтр Калмана)
и линейно квадратичный гауссовский регулятор.


\section{Постановка задачи}
Дан объект управления:
$$
\begin{cases}
    \dot{x}=Ax+bu+Gw,\ x(0),\\
    y=Cx+\nu,
\end{cases}
$$
где $w,\nu$ -- сигналы вида <<белый шум>>
с нулевыми математическими
ожиданиями $M[w]=M[\nu]=0$ и
автокорреляционными функциями
$M[w(t)w^T(\tau)]=W\delta(t-\tau),M[\nu(t)\nu(\tau)]=V\delta(t-\tau)$
с известными постоянными
спектральными плотностями (энергиями) $W$ и $V$ соответственно.


Задача заключается в построении оптимального наблюдателя,
генерирующего оценку $\hat{x}$:
$$
||x(t)-\hat{x}(t)||\leq\Delta\,\forall t\geq T,
$$
где $\Delta$ и $T$ -- максимальная ошибка и время настройки наблюдателя
соответственно. Критерий оптимальности представлен следующим
функционалом:
$$
J=M[e_L^Te_L],
$$
где $e_L=x-\hat{x}$ -- ошибка наблюдения, $M[\cdot]$ -- математическое ожидание.


Наблюдатель задается следующей структурой:
$$
\dot{\hat{x}}=A\hat{x}+bu+L(y-C\hat{x}),\ \hat{x}(0),
$$
где матрица $L$ рассчитывается на основе уравнения Риккати:
$$
\begin{cases}
    AP+PA^T+GWG^T-PC^TV^{-1}CP=0,\\
    L=PC^TV^{-1}
\end{cases}
$$


\section{Экспериментальная часть}
\subsection{Исходные данные}
Согласно варианту 31, матрицы $A,b,C,G,W,Q$:
$$
A=\begin{bmatrix}
    0&-9\\ 1& -4
\end{bmatrix},\ b=\begin{bmatrix}
    5\\ 0
\end{bmatrix},\
C=\begin{bmatrix}
    1\\0
\end{bmatrix}^T,\
G=\begin{bmatrix}
    1&0\\0&1
\end{bmatrix},\
W=\begin{bmatrix}
    7&5\\5&6
\end{bmatrix},\
Q=\begin{bmatrix}
    3&0\\ 0&4
\end{bmatrix}
$$
Параметры $V=1,r=4$.


\subsection{Расчет матрицы коррекции наблюдателя}
Решим в матлабе уравнение Риккати, получим:
$$
P=\begin{bmatrix}
    2.3813    &0.0739\\
    0.0739    &0.7678
\end{bmatrix},\ L=\begin{bmatrix}
    2.3813\\
    0.0739
\end{bmatrix}
$$
Собственные числа замкнутой системы:
$$
\sigma\left( A-LC \right)=\left\{ -3.1907 \pm 2.7713i\right\}
$$
Замкнутая система асимптотически устойчива.


\subsection{Моделирование системы}
Схема моделирования системы представлена на рис. (\ref{fig:sch}).


Начальные условия:
\begin{lstlisting}[label=init, caption={Начальные условия схемы моделирования}]
set_param('sim1/A', 'Gain', mat2str(A));
set_param('sim1/A1', 'Gain', mat2str(A));
set_param('sim1/b', 'Gain', mat2str(b));
set_param('sim1/b1', 'Gain', mat2str(b));
set_param('sim1/C', 'Gain', mat2str(C));
set_param('sim1/C1', 'Gain', mat2str(C));
set_param('sim1/L', 'Gain', mat2str(L));
set_param('sim1/G', 'Gain', mat2str(G));
set_param('sim1/W', 'Gain', mat2str(W));
\end{lstlisting}


\begin{figure}[H]
    \centering
    \includegraphics[scale=0.5]{sch.png}
    \caption{Схема моделирования системы}
    \label{fig:sch}
\end{figure}


Промоделируем систему при $u=\sin{t},x(0)=\left[ 1,0 \right]^T,\hat{x}(0)=\left[ 0,0 \right]^T$:
\begin{figure}[H]
    \centering
    \includegraphics[scale=0.9]{1elx.png}
    \caption{Ошибка наблюдения $e_L=x-\hat{x}$ при $L$}
    \label{fig:1elx}
\end{figure}
\begin{figure}[H]
    \centering
    \includegraphics[scale=0.9]{1J.png}
    \caption{Критерий качества при $L$}
    \label{fig:1J}
\end{figure}


Среднее значение критерия качества $M[J]=23.3316$.


\subsection{Моделирование системы с отклонениями коэффициентов в матрице коррекции наблюдателя}
Отклоним коэффициенты матрицы коррекции наблюдателя:
$$
L_b=1.2L=\begin{bmatrix}
    3.4577\\
    0.1072
\end{bmatrix}
$$
Спектр замкнутой системы:
$$
\sigma\left( A-L_bC \right)=\left\{ -3.7288 \pm 2.8216i\right\}
$$
Замкнутая система осталась асимптотически устойчивой.


Проведем аналогичное моделирование при $L_b$:
\begin{figure}[H]
    \centering
    \includegraphics[scale=0.9]{2elx.png}
    \caption{Ошибка наблюдения $e_L=x-\hat{x}$ при $L_b=1.2L$}
    \label{fig:2elx}
\end{figure}
\begin{figure}[H]
    \centering
    \includegraphics[scale=0.9]{2J.png}
    \caption{Критерий качества при $L_b=1.2L$}
    \label{fig:2J}
\end{figure}


Среднее значение критерия качества $M[J]=20.2593$.
По сравнению с эталоном значение уменьшилось на $\approx13.17\%$.


Проверим $L_b=0.8L=\left[ 1.905,0.0591 \right]^T$:
\begin{figure}[H]
    \centering
    \includegraphics[scale=0.9]{3elx.png}
    \caption{Ошибка наблюдения $e_L=x-\hat{x}$ при $L_b=0.8L$}
    \label{fig:3elx}
\end{figure}
\begin{figure}[H]
    \centering
    \includegraphics[scale=0.9]{3J.png}
    \caption{Критерий качества при $L_b=0.8L$}
    \label{fig:3J}
\end{figure}


Среднее значение критерия качества $M[J]=25.4642$ увеличилось.


\subsection{Моделирование системы с отклонениями значений в матрице энергии W}
Отклоним значения в матрице $W$:
$$
W_b=0.5W=\begin{bmatrix}
    3.5	&2.5\\
    2.5	&3
\end{bmatrix}\Rightarrow L=\begin{bmatrix}
    1.4575\\
0.0764
\end{bmatrix}
$$
Собственные числа $W_b:\left\{ 0.3688, 2.8812 \right\}$.


Матрица $W_b$ симметрична и ее спектр не содержит неположительных чисел,
следовательно матрица $W_b$ положительно определена.


Проведем аналогичное моделирование системы:
\begin{figure}[H]
    \centering
    \includegraphics[scale=0.9]{4elx.png}
    \caption{Ошибка наблюдения $e_L=x-\hat{x}$ при $W_b=0.5W$}
    \label{fig:4elx}
\end{figure}
\begin{figure}[H]
    \centering
    \includegraphics[scale=0.9]{4J.png}
    \caption{Критерий качества при $W_b=0.5W$}
    \label{fig:4J}
\end{figure}


Среднее значение $M[J]=7.4669$ значительно уменьшилось,
ошибка наблюдения уменьшилась.


\subsection{Моделирование системы с отклонением значения в матрице энергии V}
Отклоним значение матрицы $V$:
$$
V_b=0.5V=0.5>0\Rightarrow L=\begin{bmatrix}
    3.6943\\
0.0196
\end{bmatrix}
$$


Проведем аналогичное моделирование системы при $V_b,W$:
\begin{figure}[H]
    \centering
    \includegraphics[scale=0.9]{5elx.png}
    \caption{Ошибка наблюдения $e_L=x-\hat{x}$ при $V_b=0.5V,W$}
    \label{fig:5elx}
\end{figure}
\begin{figure}[H]
    \centering
    \includegraphics[scale=0.9]{5J.png}
    \caption{Критерий качества при $V_b=0.5V,W$}
    \label{fig:5J}
\end{figure}


Критерий качества в среднем $M[J]=18.6695$ и ошибка
уменьшились менее значимо, чем в случае с $W_b=0.5W$.


Промоделируем систему при $V_b=0.5V,W_b=0.5W,L=\left[ 2.3813,0.0739 \right]^T$:
\begin{figure}[H]
    \centering
    \includegraphics[scale=0.9]{6elx.png}
    \caption{Ошибка наблюдения $e_L=x-\hat{x}$ при $V_b=0.5V,W_b=0.5W$}
    \label{fig:6elx}
\end{figure}
\begin{figure}[H]
    \centering
    \includegraphics[scale=0.9]{6J.png}
    \caption{Критерий качества при $V_b=0.5V,W_b=0.5W$}
    \label{fig:6J}
\end{figure}


Среднее значения критерия качества $M[J]=6.0945$
получилось наименьшим из всех по сравнению с предыдущими пунктами
-- мы уменьшили влияния обеих помех. С ошибкой наблюдения аналогично.


\subsection{Моделирование системы с ЛКГ регулятором и фильтром Калмана}
Система замкнута регулятором $u=-K\hat{x}$. Оценка вектора состояния
формируется на базе фильтра Калмана. Для расчета $K$ используем LQR:
$$
P=\begin{bmatrix}
0.56854&-0.4901\\-0.4901&1.415
\end{bmatrix} ,\ K=\begin{bmatrix}
    0.7106   &-0.6126
\end{bmatrix}
$$
Спектр замкнутой системы:
$$
\sigma\left( A-BK \right)=\left\{ -3.7765 \pm 2.4264i \right\}
$$
Замкнутая система асимптотически устойчива.


Схема моделирования системы:
\begin{figure}[H]
    \centering
    \includegraphics[scale=0.6]{sch2.png}
    \caption{Схема моделирования системы с ЛКГ и фильтром Калмана}
    \label{fig:sch2}
\end{figure}


Проведем аналогичное моделирование:
\begin{figure}[H]
    \centering
    \includegraphics[scale=0.9]{7elx.png}
    \caption{Ошибка наблюдения $e_L=x-\hat{x}$ при $u=-K\hat{x}$}
    \label{fig:7elx}
\end{figure}
\begin{figure}[H]
    \centering
    \includegraphics[scale=0.9]{7J.png}
    \caption{Критерий качества при $u=-K\hat{x}$}
    \label{fig:7J}
\end{figure}


Среднее значение критерия качества такое же, как в пункте 3.3: $M[J]=23.3316$.
Это потому, что динамика ошибки оценивания определяется только
параметрами наблюдателя (матрица коррекции $L$ -- замкнутая система $A-LC$) и характеристиками помех $W,V$.
Изменив управление, но не поменяв $W,V$, результат остался такой же. Управление
$u$ сокращается в уравнениях ошибки.


\section{Вывод}
В ходе выполнения лабораторной работы
был синтезирован наблюдатель (фильтр Калмана).
Было проведено моделирование системы с полученным наблюдателем.
Результаты показали, что наблюдатель синтезирован корректно
-- он обеспечивает минимальную ошибку при заданных
интенсивностях помех. При отклонении коэффициентов
матрицы коррекции наблюдателя на $+20\%$ наблюдатель
стал работать эффективнее -- значения собственных чисел
замкнутой системы уменьшились. При отклонении на $-20\%$
наблюдатель стал работать хуже. При уменьшении влияния
помех наблюдатель работает эффективнее. Управление
не влияет на эффективность наблюдения.


\appendix
\renewcommand{\thesection}{\Asbuk{section}}
\section{Приложение}
\begin{lstlisting}[label=code,caption={Программа для поиска $L,K$}]
%% plant parameters
A=[0 -9;
   1 -4];
b=[5;
    0];
C=[1 0];
G=eye(2);
W=[7 5;
    5 6];
Q=[3 0;
    0 4];
V=1;
r=4;

%% Riccati L
GWGT=G*W*G';
[P,L,e]=icare(A',C',GWGT,V);
P
L=P*C'*V^-1
e=eig(A-L*C)

%% Riccati K
v=1;
[PK,K,e]=icare(A,sqrt(v)*b,Q,r);
PK
K=inv(r)*b'*PK
eK=eig(A-b*K)
\end{lstlisting}
\end{document}