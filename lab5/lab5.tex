\documentclass[a4paper,14pt]{extarticle}

\usepackage[T2A]{fontenc}
\usepackage[utf8]{inputenc}
\usepackage[english, russian]{babel}

\usepackage[left=30mm, right=10mm, top=20mm, bottom=20mm]{geometry}

\usepackage{tempora}
\usepackage{setspace}
\onehalfspacing

\usepackage{titlesec}
\titleformat{\section}[block]{\bfseries\centering\MakeUppercase}{\thesection.}{1em}{}
\titleformat{\subsection}[block]{\bfseries}{\thesubsection.}{1em}{}
\titleformat{\subsubsection}[block]{\bfseries}{\thesubsubsection.}{1em}{}

\renewcommand{\contentsname}{\hfill \textbf{СОДЕРЖАНИЕ} \hfill\null}

\usepackage{indentfirst}
\setlength{\parindent}{1.25cm}

\usepackage{amsmath, amsfonts, amssymb}
\usepackage{graphicx}
\usepackage{caption}
\usepackage{subcaption}
\usepackage{float}
\usepackage{tikz}
\usetikzlibrary{patterns}
\usepackage{cmap}
\usepackage{hyperref}
\usepackage{xcolor}
\usepackage{listings}

\definecolor{LightGray}{gray}{0.7}

\lstdefinestyle{code}{
    language=matlab, % change if needed
    basicstyle=\small\ttfamily,
    numbers=left,
    numberstyle=\small\color{LightGray},
    stepnumber=1,
    numbersep=5pt,
    backgroundcolor=\color{white},
    showspaces=false,
    showstringspaces=false,
    showtabs=false,
    tabsize=4,
    captionpos=b,
    breaklines=true,
    breakatwhitespace=false,
    frame=single,
    rulecolor=\color{LightGray},
    linewidth=\linewidth,
    keywordstyle=\color{blue}\bfseries,
    commentstyle=\color{green!40!black},
    stringstyle=\color{violet},
    escapeinside={\%*}{*)},
    xleftmargin=10pt,
    xrightmargin=10pt,
    framexleftmargin=0pt,
    framexrightmargin=0pt
}
\lstset{style=code}

\hypersetup{
    colorlinks=true,
    linkcolor=blue,
    filecolor=magenta,
    urlcolor=cyan,
    pdftitle={lab arc},
    pdfauthor={Rumyantsev Alexey},
    pdfsubject={control},
    pdfkeywords={LaTeX, PDF},
    pdfpagemode=FullScreen,
}

\graphicspath{{src/images/}}

\begin{document}

\begin{titlepage}
    \begin{center}
        МИНИСТЕРСТВО НАУКИ И ВЫСШЕГО ОБРАЗОВАНИЯ РОССИЙСКОЙ ФЕДЕРАЦИИ\\
        \vspace*{2.5mm}
        Федеральное государственное автономное образовательное учреждение высшего образования
        «НАЦИОНАЛЬНЫЙ ИССЛЕДОВАТЕЛЬСКИЙ УНИВЕРСИТЕТ ИТМО»\\
        \vspace*{2.5mm}
        ФАКУЛЬТЕТ СИСТЕМ УПРАВЛЕНИЯ И РОБОТОТЕХНИКИ
        \vfill

        {\large\bfseries ОТЧЕТ ПО ЛАБОРАТОРНОЙ РАБОТЕ №5}\\
        {\large по дисциплине}\\
        {\large«ТЕОРИЯ ОПТИМАЛЬНОГО УПРАВЛЕНИЯ»}\\
        {\large на тему}\\
        {\large «СИНТЕЗ ОПТИМАЛЬНОГО НАБЛЮДАТЕЛЯ (ФИЛЬТРА КАЛМАНА) И ЛКГ-СИНТЕЗ»}\\
        Вариант 31
        \vfill

        \begin{flushright}
            Выполнил: студент гр. R3441\\
            Румянцев А. А.\medskip\\

            Проверил: преподаватель\\
            Парамонов А. В.
        \end{flushright}
        \vfill

        Санкт-Петербург\\
        2025
    \end{center}
\end{titlepage}

\setcounter{page}{2}
\tableofcontents
\newpage


\section{Цель работы}
Исследовать оптимальный наблюдатель (фильтр Калмана)
и линейно квадратичный гауссовский регулятор.


\section{Постановка задачи}
Дан объект управления:
$$
\begin{cases}
    \dot{x}=Ax+bu+Gw,\ x(0),\\
    y=Cx+\nu,
\end{cases}
$$
где $w,\nu$ -- сигналы вида <<белый шум>>
с нулевыми математическими
ожиданиями $M[w]=M[\nu]=0$ и
автокорреляционными функциями
$M[w(t)w^T(\tau)]=W\delta(t-\tau),M[\nu(t)\nu(\tau)]=V\delta(t-\tau)$
с известными постоянными
спектральными плотностями (энергиями) $W$ и $V$ соответственно.


Задача заключается в построении оптимального наблюдателя,
генерирующего оценку $\hat{x}$:
$$
||x(t)-\hat{x}(t)||\leq\Delta\,\forall t\geq T,
$$
где $\Delta$ и $T$ -- максимальная ошибка и время настройки наблюдателя
соответственно. Критерий оптимальности представлен следующим
функционалом:
$$
J=M[e_L^Te_L],
$$
где $e_L=x-\hat{x}$ -- ошибка наблюдения, $M[\cdot]$ -- математическое ожидание.


Наблюдатель задается следующей структурой:
$$
\dot{\hat{x}}=A\hat{x}+bu+L(y-C\hat{x}),\ \hat{x}(0),
$$
где матрица $L$ рассчитывается на основе уравнения Риккати:
$$
\begin{cases}
    AP+PA^T+GWG^T-PC^TV^{-1}CP=0,\\
    L=PC^TV^{-1}
\end{cases}
$$


\section{Экспериментальная часть}
\subsection{Исходные данные}
Согласно варианту 31, матрицы $A,b,C,G,W,Q$:
$$
A=\begin{bmatrix}
    0&-9\\ 1& -4
\end{bmatrix},\ b=\begin{bmatrix}
    5\\ 0
\end{bmatrix},\
C=\begin{bmatrix}
    1\\0
\end{bmatrix}^T,\
G=\begin{bmatrix}
    1&0\\0&1
\end{bmatrix},\
W=\begin{bmatrix}
    7&5\\5&6
\end{bmatrix},\
Q=\begin{bmatrix}
    3&0\\ 0&4
\end{bmatrix}
$$
Параметры $V=1,r=4$.


\subsection{Расчет матрицы коррекции наблюдателя}
...


\subsection{Моделирование системы}
...


\subsection{Моделирование системы с отклонениями коэффициентов в матрице коррекции наблюдателя}
...


\subsection{Моделирование системы с отклонениями значений в матрице энергии W}
...


\subsection{Моделирование системы с отклонением значения энергии V}
...


\subsection{Моделирование системы с ЛКГ регулятором и фильтром Калмана}
...


\section{Вывод}
...
\end{document}