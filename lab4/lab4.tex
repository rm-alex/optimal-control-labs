\documentclass[a4paper,14pt]{extarticle}

\usepackage[T2A]{fontenc}
\usepackage[utf8]{inputenc}
\usepackage[english, russian]{babel}

\usepackage[left=30mm, right=10mm, top=20mm, bottom=20mm]{geometry}

\usepackage{tempora}
\usepackage{setspace}
\onehalfspacing

\usepackage{titlesec}
\titleformat{\section}[block]{\bfseries\centering\MakeUppercase}{\thesection.}{1em}{}
\titleformat{\subsection}[block]{\bfseries}{\thesubsection.}{1em}{}
\titleformat{\subsubsection}[block]{\bfseries}{\thesubsubsection.}{1em}{}

\renewcommand{\contentsname}{\hfill \textbf{СОДЕРЖАНИЕ} \hfill\null}

\usepackage{indentfirst}
\setlength{\parindent}{1.25cm}

\usepackage{amsmath, amsfonts, amssymb}
\usepackage{graphicx}
\usepackage{caption}
\usepackage{subcaption}
\usepackage{float}
\usepackage{tikz}
\usetikzlibrary{patterns}
\usepackage{cmap}
\usepackage{hyperref}
\usepackage{xcolor}
\usepackage{listings}

\definecolor{LightGray}{gray}{0.7}

\lstdefinestyle{code}{
    language=matlab, % change if needed
    basicstyle=\small\ttfamily,
    numbers=left,
    numberstyle=\small\color{LightGray},
    stepnumber=1,
    numbersep=5pt,
    backgroundcolor=\color{white},
    showspaces=false,
    showstringspaces=false,
    showtabs=false,
    tabsize=4,
    captionpos=b,
    breaklines=true,
    breakatwhitespace=false,
    frame=single,
    rulecolor=\color{LightGray},
    linewidth=\linewidth,
    keywordstyle=\color{blue}\bfseries,
    commentstyle=\color{green!40!black},
    stringstyle=\color{violet},
    escapeinside={\%*}{*)},
    xleftmargin=10pt,
    xrightmargin=10pt,
    framexleftmargin=0pt,
    framexrightmargin=0pt
}
\lstset{style=code}

\hypersetup{
    colorlinks=true,
    linkcolor=blue,
    filecolor=magenta,
    urlcolor=cyan,
    pdftitle={lab arc},
    pdfauthor={Rumyantsev Alexey},
    pdfsubject={control},
    pdfkeywords={LaTeX, PDF},
    pdfpagemode=FullScreen,
}

\graphicspath{{src/images/}}

\begin{document}

\begin{titlepage}
    \begin{center}
        МИНИСТЕРСТВО НАУКИ И ВЫСШЕГО ОБРАЗОВАНИЯ РОССИЙСКОЙ ФЕДЕРАЦИИ\\
        \vspace*{2.5mm}
        Федеральное государственное автономное образовательное учреждение высшего образования
        «НАЦИОНАЛЬНЫЙ ИССЛЕДОВАТЕЛЬСКИЙ УНИВЕРСИТЕТ ИТМО»\\
        \vspace*{2.5mm}
        ФАКУЛЬТЕТ СИСТЕМ УПРАВЛЕНИЯ И РОБОТОТЕХНИКИ
        \vfill

        {\large\bfseries ОТЧЕТ ПО ЛАБОРАТОРНОЙ РАБОТЕ №4}\\
        {\large по дисциплине}\\
        {\large«ТЕОРИЯ ОПТИМАЛЬНОГО УПРАВЛЕНИЯ»}\\
        {\large на тему}\\
        {\large «СИНТЕЗ ОПТИМАЛЬНОГО УПРАВЛЕНИЯ. МЕТОД ДИНАМИЧЕСКОГО ПРОГРАММИРОВАНИЯ БЕЛЛМАНА»}\\
        Вариант 31
        \vfill

        \begin{flushright}
            Выполнил: студент гр. R3441\\
            Румянцев А. А.\medskip\\

            Проверил: преподаватель\\
            Парамонов А. В.
        \end{flushright}
        \vfill

        Санкт-Петербург\\
        2025
    \end{center}
\end{titlepage}

\setcounter{page}{2}
\tableofcontents
\newpage


\section{Цель работы}
Исследовать оптимальный регулятор,
синтезированный с помощью метода динамического
программирования Беллмана.


\section{Постановка задачи}
Дан линейный объект управления:
$$
\dot{x}=Ax+bu,\ x(0)
$$
и критерий качества:
$$
J=\int\limits_{0}^{\infty}x^T(\tau)Qx(\tau)+ru^2(\tau)\,d\tau
$$
Необходимо:
\begin{enumerate}
    \item Построить оптимальный регулятор с помощью метода динамического
    программирования Беллмана и промоделировать его работу на заданном
    интервале времени. Начальные условия $x(0)=\left[ 1,0 \right]^T$. Построить графики $u,x,J$;
    \item Построить критерий при отклонениях параметров регулятора от
    оптимальных значений.
\end{enumerate}


\section{Экспериментальная часть}
\subsection{Исходные данные}
Согласно варианту 31, матрицы $A,b,Q$:
$$
A=\begin{bmatrix}
    0&1\\ -2&4
\end{bmatrix},\
b=\begin{bmatrix}
    8\\ 9
\end{bmatrix},\
Q=\begin{bmatrix}
    3&0\\ 0&4
\end{bmatrix}
$$
Параметр $r=4$.


\subsection{Синтез оптимального управления}
Принцип Беллмана:
$$
J(x_0)=\min\limits_{u}{\left\{ \int\limits_{0}^{dt}L(x,u)\,dt+J(x(dt)) \right\}}
$$
Предел при $dt\to0$:
$$
J(x)=\min\limits_{u}{\left\{ L(x,u)\,dt+J(x)+dt\, \nabla J(x)f(x,u)+o(dt) \right\}}
$$
После преобразований уравнение Беллмана:
$$
\min\limits_{u}{\left\{L(x,u)+\nabla V(x)^Tf(x,u)\right\}}=0,
$$
где $L(x,u)$ -- мгновенная стоимость (интегранд в функционале $J$).


Функция Беллмана $V$ должна быть такой, чтобы при оптимальном управлении выражение под минимумом обращалось в нуль.


Тогда, при линейной системе и квадратичном критерии квадратичная форма функционала ценности:
$$
V(x)=x^TPx>0\,\forall x\neq0,\ \nabla V=2Px,
$$
где $P=P^T$ -- искомая матрица.


Условие минимума по $u$ в уравнении
Гамильтона–Якоби–Беллмана:
$$
\min\limits_{u}{\left\{ x^TQx+ru^2+\nabla V(x)\left( Ax+bu \right) \right\}}=0
$$
Подставим $V(x)$ и $\nabla V(x)$:
$$
\frac{d}{du}\left( ru^2+2x^TPbu \right)=0\Rightarrow 2ru+2b^TPx=0
$$
Выразим $u$:
$$
u(t)=-r^{-1}b^TPx(t)=-Kx(t),\ K=r^{-1}b^TP
$$
Найдем $u^2(t)$:
$$
u^2=x^TP^Tbr^{-2}b^TPx
$$
Подтавим в условие минимума по $u$ выражения $\nabla V(x), u^2(t)$:
$$
x^TQx+x^TP^Tbr^{-1}b^TPx+(2Px)^T\left( Ax+bu \right)=0,
$$
$$
x^TQx+x^TP^Tbr^{-1}b^TPx+2x^TPAx+2x^TPbu=0
$$
Найдем $2x^TPbu$:
$$
2x^TPbu=2x^TPb\left( -r^{-1}b^TPx \right)=-2x^TPbr^{-1}b^TPx
$$
Тогда, $ru^2+2x^TPbu$:
$$
ru^2+2x^TPbu=x^TPbr^{-1}b^TPx-2x^TPbr^{-1}b^TPx=-x^TPbr^{-1}b^TPx
$$
Подставим:
$$
x^TQx+2x^TPAx-x^TPbr^{-1}b^TPx=0
$$
Перепишем $2x^TPAx$:
$$
2x^TPAx=x^T\left( PA+A^TP \right)x
$$
Подставим $2x^TPAx$ и вынесем $x^T,x$:
$$
x^T\left( Q+PA+A^TP-Pbr^{-1}b^TP \right)x=0
$$
Данное условие должно выполняться для всех $x$,
следовательно подматрица:
$$
A^TP+PA-Pbr^{-1}b^TP+Q=0
$$
Получили непрерывное алгебраическое уравнение Риккати.


Уравнение Риккати -- частный случай уравнения Беллмана для LQR.


Решим уравнение Риккати, получим:
$$
P\approx\begin{bmatrix}
    40.795  &-39.0667\\
  -39.0667   &38.101
\end{bmatrix},\ K=\begin{bmatrix}
    -6.31    &7.594
\end{bmatrix}
$$
Собственные числа замкнутой системы:
$$
\sigma\left( A-BK \right)=\left\{ -2.1961,-11.6695 \right\}
$$
Замкнутая система асимптотически устойчива.


\subsection{Моделирование системы}
Схема моделирования:
\begin{figure}[H]
    \centering
    \includegraphics[scale=0.6]{sch.png}
    \caption{Схема моделирования замкнутой системы, $u=-Kx$}
    \label{fig:sch}
\end{figure}


Построим графики $x,u,J$:
\begin{figure}[H]
    \centering
    \includegraphics[scale=0.8]{1x.png}
    \caption{Вектор состояния объекта $x(t)$, $u=-Kx$}
    \label{fig:1x}
\end{figure}
\begin{figure}[H]
    \centering
    \includegraphics[scale=0.8]{1u.png}
    \caption{Управление $u=-Kx$}
    \label{fig:1u}
\end{figure}
\begin{figure}[H]
    \centering
    \includegraphics[scale=0.8]{1J.png}
    \caption{Критерий качества $J(t)$, $u=-Kx$}
    \label{fig:1J}
\end{figure}


Установившееся значение критерия качества на
интервале моделирования составило $J=40.795$.


\subsection{Критерий при отклонениях параметров регулятора}
Отклоним расчетные значения $K$ на $10\%$:
$$
K_b=1.1K=\begin{bmatrix}
    -6.941 &8.3534
\end{bmatrix}
$$
Спектр замкнутой системы:
$$
\sigma\left( A-BK \right)=\left\{ -2.0589,-13.5937 \right\}
$$
Замкнутая система асимптотически устойчива.


\subsection{Моделирование системы с отклонением}
Промоделируем систему аналогично предыдущему пункту:
\begin{figure}[H]
    \centering
    \includegraphics[scale=0.8]{2x.png}
    \caption{Вектор состояния объекта $x(t)$, $u=-K_bx$}
    \label{fig:2x}
\end{figure}
\begin{figure}[H]
    \centering
    \includegraphics[scale=0.8]{2u.png}
    \caption{Управление $u=-K_bx$}
    \label{fig:2u}
\end{figure}
\begin{figure}[H]
    \centering
    \includegraphics[scale=0.8]{2J.png}
    \caption{Критерий качества $J(t)$, $u=-K_bx$}
    \label{fig:2J}
\end{figure}


Установившееся значение критерия качества на
интервале моделирования составило $J=40.85$.


С отклоненными параметрами на стабилизацию
системы затрачивается несколько больше управления.


Критерий $J_b=40.85$ отклонился от эталонного $J=40.795$ на $+0.1348\%$,
регулятор остался достаточно эффективным, но менее
эффективным, чем в эталонном случае.


\section{Вывод}
В ходе выполнения лабораторной работы
был синтезирован оптимальный регулятор
методом динамического программирования
Беллмана. В результате получилось
алгебраическое уравнение Риккати.
Были промоделированы эталонный регулятор
и с отклоненными на $10\%$ коэффициентами.
Результаты моделирования показали
корректность проведенных расчетов и 
достаточную эффективность обоих регуляторов.
\end{document}