\documentclass[a4paper,14pt]{extarticle}

\usepackage[T2A]{fontenc}
\usepackage[utf8]{inputenc}
\usepackage[english, russian]{babel}

\usepackage[left=30mm, right=10mm, top=20mm, bottom=20mm]{geometry}

\usepackage{tempora}
\usepackage{setspace}
\onehalfspacing

\usepackage{titlesec}
\titleformat{\section}[block]{\bfseries\centering\MakeUppercase}{\thesection.}{1em}{}
\titleformat{\subsection}[block]{\bfseries}{\thesubsection.}{1em}{}
\titleformat{\subsubsection}[block]{\bfseries}{\thesubsubsection.}{1em}{}

\renewcommand{\contentsname}{\hfill \textbf{СОДЕРЖАНИЕ} \hfill\null}

\usepackage{indentfirst}
\setlength{\parindent}{1.25cm}

\usepackage{amsmath, amsfonts, amssymb}
\usepackage{graphicx}
\usepackage{caption}
\usepackage{subcaption}
\usepackage{float}
\usepackage{tikz}
\usetikzlibrary{patterns}
\usepackage{cmap}
\usepackage{hyperref}
\usepackage{xcolor}
\usepackage{listings}

\definecolor{LightGray}{gray}{0.7}

\lstdefinestyle{code}{
    language=matlab, % change if needed
    basicstyle=\small\ttfamily,
    numbers=left,
    numberstyle=\small\color{LightGray},
    stepnumber=1,
    numbersep=5pt,
    backgroundcolor=\color{white},
    showspaces=false,
    showstringspaces=false,
    showtabs=false,
    tabsize=4,
    captionpos=b,
    breaklines=true,
    breakatwhitespace=false,
    frame=single,
    rulecolor=\color{LightGray},
    linewidth=\linewidth,
    keywordstyle=\color{blue}\bfseries,
    commentstyle=\color{green!40!black},
    stringstyle=\color{violet},
    escapeinside={\%*}{*)},
    xleftmargin=10pt,
    xrightmargin=10pt,
    framexleftmargin=0pt,
    framexrightmargin=0pt
}
\lstset{style=code}

\hypersetup{
    colorlinks=true,
    linkcolor=blue,
    filecolor=magenta,
    urlcolor=cyan,
    pdftitle={lab arc},
    pdfauthor={Rumyantsev Alexey},
    pdfsubject={control},
    pdfkeywords={LaTeX, PDF},
    pdfpagemode=FullScreen,
}

\graphicspath{{src/images/}}

\begin{document}

\begin{titlepage}
    \begin{center}
        МИНИСТЕРСТВО НАУКИ И ВЫСШЕГО ОБРАЗОВАНИЯ РОССИЙСКОЙ ФЕДЕРАЦИИ\\
        \vspace*{2.5mm}
        Федеральное государственное автономное образовательное учреждение высшего образования
        «НАЦИОНАЛЬНЫЙ ИССЛЕДОВАТЕЛЬСКИЙ УНИВЕРСИТЕТ ИТМО»\\
        \vspace*{2.5mm}
        ФАКУЛЬТЕТ СИСТЕМ УПРАВЛЕНИЯ И РОБОТОТЕХНИКИ
        \vfill

        {\large\bfseries ОТЧЕТ ПО ЛАБОРАТОРНОЙ РАБОТЕ №2}\\
        {\large по дисциплине}\\
        {\large«ТЕОРИЯ ОПТИМАЛЬНОГО УПРАВЛЕНИЯ»}\\
        {\large на тему}\\
        {\large «СИНТЕЗ ОПТИМАЛЬНОГО УПРАВЛЕНИЯ. ПРИНЦИП МАКСИМУМА»}\\
        Вариант 31
        \vfill

        \begin{flushright}
            Выполнил: студент гр. R3441\\
            Румянцев А. А.\medskip\\

            Проверил: преподаватель\\
            Парамонов А. В.
        \end{flushright}
        \vfill

        Санкт-Петербург\\
        2025
    \end{center}
\end{titlepage}

\setcounter{page}{2}
\tableofcontents
\newpage


\section{Цель работы}
Исследовать метод динамической оптимизации
и синтезировать оптимальное управление,
обеспечивающее минимум функционала качества.


\section{Постановка задачи}
Дан объект, критерий, начальные условия и ограничения.
Необходимо:
\begin{enumerate}
    \item Построить оптимальный в смысле заданного критерия регулятор и
        промоделировать его работу на заданном интервале времени;
    \item Построить графики управления, переменных состояния и критерия;
    \item Рассчитать критерий при отклонениях параметров регулятора от
        оптимальных значений.
\end{enumerate}


\section{Экспериментальная часть}
\subsection{Исходные данные}
Согласно варианту 31, объект:
$$
\begin{cases}
    \dot{x}_1=x_2,\\
    \dot{x}_2=-8x_1+u
\end{cases}
$$
Критерий:
$$
J=\int\limits_{0}^{3}u^2(\tau)\,d\tau
$$
Начальные условия:
$$
x_1(0)=x_2(0)=0
$$
Ограничения:
$$
x_1(3)=5,\ x_2(3)=0
$$


\subsection{Синтез оптимального управления}
Гамильтониан:
$$
H=\varphi_0u^2+\varphi\dot{x},
$$
где $\varphi_i$ -- динамические множители Лагранжа.
Примем $\varphi_0=-1$. Так как у объекта два состояния,
Гамильтониан:
$$
H=-u^2+\varphi_1\dot{x}_1+\varphi_2\dot{x}_2=-u^2+\varphi_1x_2+\varphi_2\left( -8x_1+u \right)
$$
Система уравнений Эйлера-Лагранжа (принцип максимума):
$$
\begin{cases}
    \dot{\varphi}_i=-\frac{\partial H}{\partial x_i},\\
    \frac{\partial H}{\partial u}=0
\end{cases}\Rightarrow\begin{cases}
    \dot{\varphi_1}=8\varphi_2,\\
    \dot{\varphi_2}=-\varphi_1,\\
    -2u+\varphi_2=0
\end{cases}\Rightarrow\begin{cases}
    \dot{\varphi_1}=8\varphi_2,\\
    \dot{\varphi_2}=-\varphi_1,\\
    u=\frac{1}{2}\varphi_2
\end{cases}
$$
Задействуем уравнение объекта $\dot{x}=Ax+Bu$:
$$
\begin{cases}
    \dot{\varphi_1}=8\varphi_2,\\
    \dot{\varphi_2}=-\varphi_1,\\
    \dot{x}_1=x_2,\\
    \dot{x}_2=-8x_1+\frac{1}{2}\varphi_2
\end{cases}
$$
В матричном виде:
$$
\begin{bmatrix}
    \dot{\varphi}_1\\
    \dot{\varphi}_2\\
    \dot{x}_1\\
    \dot{x}_2
\end{bmatrix}=\begin{bmatrix}
    0 &8 &0 &0\\
    -1 &0 &0 &0\\
    0 &0 &0 &1\\
    0 &0.5 &-8 &0
\end{bmatrix}\begin{bmatrix}
    \varphi_1\\
    \varphi_2\\
    x_1\\
    x_2
\end{bmatrix}
$$
\end{document}