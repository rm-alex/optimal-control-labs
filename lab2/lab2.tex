\documentclass[a4paper,14pt]{extarticle}

\usepackage[T2A]{fontenc}
\usepackage[utf8]{inputenc}
\usepackage[english, russian]{babel}

\usepackage[left=30mm, right=10mm, top=20mm, bottom=20mm]{geometry}

\usepackage{tempora}
\usepackage{setspace}
\onehalfspacing

\usepackage{titlesec}
\titleformat{\section}[block]{\bfseries\centering\MakeUppercase}{\thesection.}{1em}{}
\titleformat{\subsection}[block]{\bfseries}{\thesubsection.}{1em}{}
\titleformat{\subsubsection}[block]{\bfseries}{\thesubsubsection.}{1em}{}

\renewcommand{\contentsname}{\hfill \textbf{СОДЕРЖАНИЕ} \hfill\null}

\usepackage{indentfirst}
\setlength{\parindent}{1.25cm}

\usepackage{amsmath, amsfonts, amssymb}
\usepackage{graphicx}
\usepackage{caption}
\usepackage{subcaption}
\usepackage{float}
\usepackage{tikz}
\usetikzlibrary{patterns}
\usepackage{cmap}
\usepackage{hyperref}
\usepackage{xcolor}
\usepackage{listings}

\definecolor{LightGray}{gray}{0.7}

\lstdefinestyle{code}{
    language=matlab, % change if needed
    basicstyle=\small\ttfamily,
    numbers=left,
    numberstyle=\small\color{LightGray},
    stepnumber=1,
    numbersep=5pt,
    backgroundcolor=\color{white},
    showspaces=false,
    showstringspaces=false,
    showtabs=false,
    tabsize=4,
    captionpos=b,
    breaklines=true,
    breakatwhitespace=false,
    frame=single,
    rulecolor=\color{LightGray},
    linewidth=\linewidth,
    keywordstyle=\color{blue}\bfseries,
    commentstyle=\color{green!40!black},
    stringstyle=\color{violet},
    escapeinside={\%*}{*)},
    xleftmargin=10pt,
    xrightmargin=10pt,
    framexleftmargin=0pt,
    framexrightmargin=0pt
}
\lstset{style=code}

\hypersetup{
    colorlinks=true,
    linkcolor=blue,
    filecolor=magenta,
    urlcolor=cyan,
    pdftitle={lab arc},
    pdfauthor={Rumyantsev Alexey},
    pdfsubject={control},
    pdfkeywords={LaTeX, PDF},
    pdfpagemode=FullScreen,
}

\graphicspath{{src/images/}}

\begin{document}

\begin{titlepage}
    \begin{center}
        МИНИСТЕРСТВО НАУКИ И ВЫСШЕГО ОБРАЗОВАНИЯ РОССИЙСКОЙ ФЕДЕРАЦИИ\\
        \vspace*{2.5mm}
        Федеральное государственное автономное образовательное учреждение высшего образования
        «НАЦИОНАЛЬНЫЙ ИССЛЕДОВАТЕЛЬСКИЙ УНИВЕРСИТЕТ ИТМО»\\
        \vspace*{2.5mm}
        ФАКУЛЬТЕТ СИСТЕМ УПРАВЛЕНИЯ И РОБОТОТЕХНИКИ
        \vfill

        {\large\bfseries ОТЧЕТ ПО ЛАБОРАТОРНОЙ РАБОТЕ №2}\\
        {\large по дисциплине}\\
        {\large«ТЕОРИЯ ОПТИМАЛЬНОГО УПРАВЛЕНИЯ»}\\
        {\large на тему}\\
        {\large «СИНТЕЗ ОПТИМАЛЬНОГО УПРАВЛЕНИЯ. ПРИНЦИП МАКСИМУМА»}\\
        Вариант 31
        \vfill

        \begin{flushright}
            Выполнил: студент гр. R3441\\
            Румянцев А. А.\medskip\\

            Проверил: преподаватель\\
            Парамонов А. В.
        \end{flushright}
        \vfill

        Санкт-Петербург\\
        2025
    \end{center}
\end{titlepage}

\setcounter{page}{2}
\tableofcontents
\newpage


\section{Цель работы}
Исследовать метод динамической оптимизации
и синтезировать оптимальное управление,
обеспечивающее минимум функционала качества.


\section{Постановка задачи}
Дан объект, критерий, начальные условия и ограничения.
Необходимо:
\begin{enumerate}
    \item Построить оптимальный в смысле заданного критерия регулятор и
        промоделировать его работу на заданном интервале времени;
    \item Построить графики управления, переменных состояния и критерия;
    \item Рассчитать критерий при отклонениях параметров регулятора от
        оптимальных значений.
\end{enumerate}


\section{Экспериментальная часть}
\subsection{Исходные данные}
Согласно варианту 31, объект:
$$
\begin{cases}
    \dot{x}_1=x_2,\\
    \dot{x}_2=-8x_1+u
\end{cases}
$$
Критерий:
$$
J=\int\limits_{0}^{3}u^2(\tau)\,d\tau
$$
Начальные условия:
$$
x_1(0)=x_2(0)=0
$$
Ограничения:
$$
x_1(3)=5,\ x_2(3)=0
$$


\subsection{Синтез оптимального управления}
Гамильтониан:
$$
H=\varphi_0u^2+\varphi\dot{x},
$$
где $\varphi_i$ -- динамические множители Лагранжа.
Примем $\varphi_0=-1$. Так как у объекта два состояния,
Гамильтониан:
$$
H=-u^2+\varphi_1\dot{x}_1+\varphi_2\dot{x}_2=-u^2+\varphi_1x_2+\varphi_2\left( -8x_1+u \right)
$$
Система уравнений Эйлера-Лагранжа (принцип максимума Понтрягина):
$$
\begin{cases}
    \dot{\varphi}_i=-\frac{\partial H}{\partial x_i},\\
    \frac{\partial H}{\partial u}=0
\end{cases}\Rightarrow\begin{cases}
    \dot{\varphi_1}=8\varphi_2,\\
    \dot{\varphi_2}=-\varphi_1,\\
    -2u+\varphi_2=0
\end{cases}\Rightarrow\begin{cases}
    \dot{\varphi_1}=8\varphi_2,\\
    \dot{\varphi_2}=-\varphi_1,\\
    u=\frac{1}{2}\varphi_2
\end{cases}
$$
Задействуем уравнение объекта $\dot{x}=Ax+Bu$:
$$
\begin{cases}
    \dot{\varphi_1}=8\varphi_2,\\
    \dot{\varphi_2}=-\varphi_1,\\
    \dot{x}_1=x_2,\\
    \dot{x}_2=-8x_1+\frac{1}{2}\varphi_2
\end{cases}
$$
В матричном виде (форма Коши):
$$
\begin{bmatrix}
    \dot{\varphi}_1\\
    \dot{\varphi}_2\\
    \dot{x}_1\\
    \dot{x}_2
\end{bmatrix}=\begin{bmatrix}
    0 &8 &0 &0\\
    -1 &0 &0 &0\\
    0 &0 &0 &1\\
    0 &0.5 &-8 &0
\end{bmatrix}\begin{bmatrix}
    \varphi_1\\
    \varphi_2\\
    x_1\\
    x_2
\end{bmatrix}
$$
Решение системы в Maple:
$$
\begin{cases}
    \varphi_1(t)=43.446\sin(2.823t) - 26.645\cos(2.823t),\\
    \varphi_2(t)=15.36\cos(2.823t) + 9.42\sin(2.823t),\\
    x_1(t)=1.358\sin(2.823t)t - 0.833\cos(2.823t)t + 0.294\sin(2.823t),\\
    x_2(t)=3.84\cos(2.823t)t + 2.355\sin(2.823t)t + 1.358\sin(2.823t)
\end{cases}
$$
Построим графики $x(t),u(t),J(t)$:
\begin{figure}[H]
    \centering
    \includegraphics[scale=0.42]{1task_0.5.png}
    \caption{Графики $x(t),u(t),J(t)$}
    \label{fig:1task_05}
\end{figure}


Система сошлась к ограничениям $x_1(3)=5,x_2(3)=0$, при этом $J(3)=126.987$.


\subsection{Критерий при отклонениях параметра}
Рассчитаем критерий и построим графики при отклонениях параметров регулятора от
оптимальных значений -- исследуем влияние отклонения коэффициента
$k$ в законе управления $u=k\varphi_2,k\in\left\{ 0.05,0.1,0.25,0.75,1 \right\}$.


Графики $x(t),u(t),J(t)$ для различных $k$:
\begin{figure}[H]
    \centering
    \includegraphics[scale=0.42]{1task_0.05.png}
    \caption{Графики $x(t),u(t),J(t)$ при $k=0.05$}
    \label{fig:1task_005}
\end{figure}
\begin{figure}[H]
    \centering
    \includegraphics[scale=0.42]{1task_0.1.png}
    \caption{Графики $x(t),u(t),J(t)$ при $k=0.1$}
    \label{fig:1task_01}
\end{figure}
\begin{figure}[H]
    \centering
    \includegraphics[scale=0.42]{1task_0.25.png}
    \caption{Графики $x(t),u(t),J(t)$ при $k=0.25$}
    \label{fig:1task_025}
\end{figure}
\begin{figure}[H]
    \centering
    \includegraphics[scale=0.42]{1task_0.75.png}
    \caption{Графики $x(t),u(t),J(t)$ при $k=0.75$}
    \label{fig:1task_075}
\end{figure}
\begin{figure}[H]
    \centering
    \includegraphics[scale=0.42]{1task_1.png}
    \caption{Графики $x(t),u(t),J(t)$ при $k=1$}
    \label{fig:1task_1}
\end{figure}


Таблица сравнения значения критерия
$$
J_k=\int\limits_{0}^{3}u^2(\tau)\,d\tau
$$
при различных $k$:
\begin{table}[htbp]
\centering
\label{tab:J_comparison}
\begin{tabular}{c c c c}
\hline
$k$ & $J(3)$ & Отклонение, \% & Примечание \\
\hline
0.05   & 127.000250 & +0.0101 & \\
0.10   & 127.000690 & +0.0104 & \\
0.25   & 126.988833 & +0.0011 & \\
\textbf{0.50} & \textbf{126.987432} & \textbf{0.0000} & \textbf{эталон (оптимальное $k$)} \\
0.75   & 127.000383 & +0.0102 & \\
1.00   & 127.000690 & +0.0104 & \\
\hline
\end{tabular}
\caption{Сравнение значения критерия $J(3)$ при различных коэффициентах усиления $k$}
\end{table}


Значение критерия отклоняется несильно от эталона при изменении коэффициента управления $k$.
Состояния сходятся в те же точки, что и эталонная модель.


\section{Вывод}
В ходе выполнения лабораторной работы
был исследован метод динамической оптимизации
-- принцип максимума Понтрягина.
Был синтезирован оптимальный в смысле
заданного критерия регулятор и промоделирована
система с ним. Результаты сошлись с ожидаемыми
значениями. Изменение коэффициента при управлении
незначительно влияет на значение критерия,
при этом состояния сходятся в те же точки, как в эталонном случае.


\appendix
\renewcommand{\thesection}{\Asbuk{section}}
\section{Приложение}
\begin{lstlisting}[label=mc,caption={Программа для решения системы}]
evalf[5](dsolve({diff(f1(t), t) = 8*f2(t), ...
diff(f2(t), t) = -f1(t), ...
diff(x1(t), t) = x2(t), ...
diff(x2(t), t) = -8*x1(t) + 0.5*f2(t), ...
x1(0) = 0, x1(3) = 5, x2(0) = 0, x2(3) = 0}, ...
{f1(t), f2(t), x1(t), x2(t)}))
\end{lstlisting}


\section{Приложение}
\begin{lstlisting}[label=code, caption={Программа для построения графиков}]
clear; clc; close all;

omega = 2.8284;

t = linspace(0, 3, 1000);

% coef = 0.5;
% phi1 = 43.446 * sin(omega*t) - 26.645 * cos(omega*t);
% phi2 = 15.360 * cos(omega*t) + 9.4202 * sin(omega*t);

% coef = 0.05;
% phi1 = 434.50 * sin(omega*t) - 266.41 * cos(omega*t);
% phi2 = 153.62 * cos(omega*t) + 94.190 * sin(omega*t);

% coef = 0.1;
% phi1 = 217.25 * sin(omega*t) - 133.21 * cos(omega*t);
% phi2 = 76.808 * cos(omega*t) + 47.098 * sin(omega*t);

% coef = 0.25;
% phi1 = 86.899 * sin(omega*t) - 53.283 * cos(omega*t);
% phi2 = 30.722 * cos(omega*t) + 18.838 * sin(omega*t);

% coef = 0.75;
% phi1 = 28.966 * sin(omega*t) - 17.762 * cos(omega*t);
% phi2 = 10.241 * cos(omega*t) + 6.2798 * sin(omega*t);

coef = 1;
phi1 = 21.725 * sin(omega*t) - 13.321 * cos(omega*t);
phi2 = 7.6808 * cos(omega*t) + 4.7098 * sin(omega*t);

u = coef * phi2;

% 0.05, 0.1, 0.25, 0.5
x1 = 1.3578 * t .* sin(omega*t) ...
         - 0.83254 * t .* cos(omega*t) ...
         + 0.29434 * sin(omega*t);

x2 = 3.8405 * t .* cos(omega*t) ...
         + 2.3548 * t .* sin(omega*t) ...
         + 1.3578 * sin(omega*t);

J = cumtrapz(t, u.^2);

figure('Position',[100,100,1200,800]);

subplot(2,2,1);
plot(t, x1, 'b-', 'LineWidth', 1.8); hold on;
plot(t, x2, 'r--', 'LineWidth', 1.8);
yline(5, '--k', 'LineWidth', 1.2);
yline(0, ':k', 'LineWidth', 1.2);

plot(3, x1(end), 'bo', ...
     'MarkerSize', 8, 'MarkerFaceColor','b');
plot(3, x2(end), 'ro', ...
     'MarkerSize', 8, 'MarkerFaceColor','r');
xlabel('t'); ylabel('x_1(t), x_2(t)');
title('State x_1(t) and x_2(t)');
legend('x_1(t)', 'x_2(t)', ...
       'x_1=5', 'x_2=0', 'Location','best');
grid on;

subplot(2,2,2);
plot(t, u, 'm-', 'LineWidth', 2);
xlabel('t'); ylabel('u(t)');
title(['Optimal control u(t) = ', ...
       num2str(coef), '\phi_2(t)']);
grid on;

subplot(2,2,3);
plot(t, J, 'k-', 'LineWidth', 2);
xlabel('t'); ylabel('J(t) = \int_0^t u^2 d\tau');
title(['Quality criteria J(t),  J(3) = ', ...
       num2str(J(end), '%.6f')]);
grid on;

subplot(2,2,4);
plot(x1, x2, 'g-', 'LineWidth', 1.8); hold on;
plot(x1(1), x2(1), 'go', ...
     'MarkerFaceColor','g', 'MarkerSize',8);
plot(x1(end), x2(end), 'ro', ...
     'MarkerFaceColor','r', 'MarkerSize',8);
text(x1(1), x2(1), '  (0,0)', ...
     'FontSize',10, 'VerticalAlignment','bottom');
text(x1(end), x2(end), ...
     sprintf('  (5,0)\n  t=%.1f', t(end)), ...
      'FontSize',10, 'VerticalAlignment','bottom');
xlabel('x_1'); ylabel('x_2');
title('Phase trajectory');
grid on; axis equal;

sgtitle('Optimal control considering Maple solution', ...
        'FontSize',14, 'FontWeight','bold');

fprintf('Final values:\n');
fprintf('x1(3) = %.6f (must be 5)\n', x1(end));
fprintf('x2(3) = %.6f (must be 0)\n', x2(end));
fprintf('u(3) = %.6f\n', u(end));
fprintf('J(3) = %.6f\n', J(end));
\end{lstlisting}
\end{document}